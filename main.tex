\documentclass[11pt,a4paper]{amsart}
\usepackage[english]{babel}
\usepackage[T1]{fontenc}
\usepackage[utf8]{inputenc}
\usepackage{lmodern}
\usepackage{amsmath,amsfonts,amssymb}
\usepackage[pdftex]{hyperref}
\usepackage{setspace}
\usepackage{color}
\setstretch{1.1}

\title{Curvature of complexified K\"{a}hler moduli}
\author{Gunnar Þór Magnússon}

\newtheorem{theo}{Theorem}[section]
\newtheorem{prop}[theo]{Proposition}
\newtheorem{coro}[theo]{Corollary}
\newtheorem{lemm}[theo]{Lemma}
\newtheorem*{theo*}{Theorem}
\theoremstyle{definition}
\newtheorem*{defi}{Definition}
\newtheorem{exam}[theo]{Example}
\theoremstyle{remark}
\newtheorem*{rema}{Remark}

\newcommand{\ZZ}{\mathbb{Z}}
\newcommand{\QQ}{\mathbb{Q}}
\newcommand{\RR}{\mathbb{R}}
\newcommand{\CC}{\mathbb{C}}
\newcommand{\PP}{\mathbb{P}}

\newcommand{\End}{\mathop{\mathrm{End}}}
\newcommand{\Aut}{\mathop{\mathrm{Aut}}}
\newcommand{\id}{\mathop{\mathrm{id}}}
\newcommand{\Vol}{\mathop{\mathrm{Vol}}}
\def\Im{\mathop{\rm Im}}

\def\ov#1{\overline{#1}}

%% Double inner product
\def\llangle{\langle\!\langle}
\def\rrangle{\rangle\!\rangle}

%% Weil--Petersson part
\def\rel{\CR^0\pi_*\smash{\Omega_{X/S}^n}}
\def\relt{\CR^0\pi_*\Omega_{\widetilde{X}/S}^n}
\def\CO{\mathcal{O}}
\def\CX{\mathcal{X}}
\def\CR{\mathcal{R}}

%% Operators
\DeclareMathOperator{\Gr}{Gr}
\DeclareMathOperator{\Hom}{Hom}
\def\d{\partial}
\def\dbar{\bar\partial}

\def\p{\times}

%% Letters
\def\cR{\mathcal{R}}
\def\cO{\mathcal{O}}
\def\cX{\mathcal{X}}

%% Cohomology groups
\def\coho#1{\mathrm{H}^{#1}}

%% Metric and associated operators
\def\met#1#2{\langle #1, \ov{#2} \rangle}
\def\Lef{\Lambda}

%% Connection and curvature
\def\chern{D}
\def\conn{\nabla}
\def\conngm{\nabla_{\mathrm{GM}}}
\def\curv{\tfrac{i}{2\pi}\Theta}

%% Intersection products with a Ahler form
\def\q#1{\frac{1}{\Vol}\int_X #1 \wedge \kf\^{n-1}}
\def\qq#1#2{\frac{1}{\Vol}\int_X #1 \wedge #2 \wedge \kf\^{n-2}}
\def\qqq#1#2#3{\frac{1}{\Vol}\int_X #1 \wedge #2 \wedge #3 \wedge \kf\^{n-3}}
\def\qqqq#1#2#3#4{\frac{1}{\Vol}\int_X #1 \wedge #2 \wedge#3 \wedge#4 \wedge \kf\^{n-4}}

%% The K\"{a}hler form and class of a Hermitian metric
\def\kf{\omega}
\def\ckf{\alpha}
\def\kcForm{\Omega}
\def\kcHerm{h_R}

%% Vertical and horizontal vectors
\def\ver#1{#1^{\mathrm{V}}}
\def\hor#1{#1^{\mathrm{H}}}

%% Tangent vectors in K\"{a}hler cone
\def\ton{u}
\def\ttw{v}
\def\tth{z}
\def\tfo{w}
\def\TKCon{\xi}
\def\TKCtw{\nu}
\def\TKCth{\eta}
\def\TKCfo{\chi}

%% Algebra power
\def\^#1{^{[#1]}}

%% K\"{a}hler cones
\def\KC{C}
\def\RKC{\mathcal{\KC}}

\begin{document}

\begin{abstract}
The complexified K\"{a}hler cone of a compact manifold carries a natural
Hermitian metric. We show that this metric is K\"{a}hler, calculate
its curvature and show that its holomorphic sectional curvature is
negative. We also give a relative version of this metric on the
K\"{a}hler cones of a family of manifolds and show that it extends to a
closed semipositive form on the total space of those cones. For families
of compact Kahler manifolds with zero first Chern class, we then obtain
a natural Kahler metric on the total space.
\end{abstract}

\maketitle



\section{On the cone of a fixed manifold}
\label{section:fixemanifold}

\subsection*{The K\"{a}hler cone}
Let $X$ be a compact K\"{a}hler manifold of dimension $\dim_{\CC} X = n$.

\begin{defi}
The \emph{K\"{a}hler cone} of $X$ is the set
\begin{equation*}
C_{\RR}(X) = \{ \kf \in \coho{1,1}(X,\RR) 
\mid
\text{$\kf$ contains a K\"{a}hler metric}
\}.
\end{equation*}
The \emph{complexified K\"{a}hler cone} of $X$ is
\begin{align*}
\KC(X) 
&:=
\{ \ckf \in \coho{1,1}(X,\CC) 
\mid
\text{$\Im\ckf$ contains a K\"{a}hler metric}
\}
\\
&\phantom{:}= \coho{1,1}(X,\RR) \oplus i C_{\RR}(X).
\end{align*}
\end{defi}

If there can be no confusion about the underlying manifold $X$, we'll
just write $C$ for its complexified K\"{a}hler cone.
As the names suggest, these are open cones in the finite-dimensional
vector spaces $\coho{1,1}(X,\RR)$ and $\coho{1,1}(X,\CC)$. The K\"{a}hler
cone is the trancendental analogue of the ample cone of a projective
variety. It is described by the the trancendental version of the 
Nakai--Moishezon criteria due to Demailly and Paun~\cite{DemaillyPaun}:


\begin{theo}
The K\"{a}hler cone of $X$ is a connected component of the set of real
$(1,1)$-cohomology classes that are numerically positive on analytic
cycles, that is, classes $\alpha$ such that $\int_{Z} a^p > 0$ for every
irreducible analytic set $Z$ in $X$ of dimension $p$.
\end{theo}


\subsection*{The K\"{a}hler metric on the cone}

Since the complexified K\"{a}hler cone $C$ is an open set in a complex
vector space, we can view it as a complex manifold in its own right.
As such, it carries a natural smooth Hermitian metric, given at each
point $\ckf$ by the inner product that the imaginary part $\Im\ckf$
defines on the cohomology of $X$.

Before we state our first result, let's agree on some notation. If
$\alpha$ is an element of an associative algebra, we write $\alpha\^k
:= \alpha^k/k!$ for all $k \geq 0$. This notation is quite convenient
for calculations with Kahler forms in the cohomology ring of $X$; I
stole it shamelessly from Georg Schumacher.


\begin{prop}
Let $\ton,\ttw$ be elements of $T_C$ at a point $\ckf$. Let $\kf =
\Im\ckf$ and let $\Lef$ be the adjoint of the Lefschetz operator that
$\kf$ defines.  The Hermitian metric on $C$ can be defined in any of the
following ways.
\begin{enumerate}
    \item 
$$
\met{\ton}{\ttw}
= \q{\ton} \cdot \q{\ov{\ttw}}
- \qq{\ton}{\ov\ttw}.
$$
    \item
\hfil
$
\met{\ton}{\ttw}
= \Lef(\ton)\Lef(\ov\ttw)
- \Lef\^2(\ton\wedge\ov\ttw).
$
\hfil
    \item
The metric is the Hermitian form associated to 
$-4\d\dbar\frac{i}{2}\Vol$, where $\Vol : C \to \RR$, $\ckf \mapsto
\Vol(X,\Im\ckf)$.
\end{enumerate}
It follows that this Hermitian metric is a K\"{a}hler metric.
\end{prop}


\begin{proof}
That $(1)$ and $(2)$ define the same Hermitian form is a simple
calculation based on the adjoint property of $\Lef$, and that $(2)$
agrees with the inner product that $\kf$ defines can be seen by taking
the primitive decomposition of $\ton$ and $\ttw$, plugging it into $(2)$
and calculating until the Hodge--Riemann bilinear relations say that we
have the correct inner product.

Note that we can view $\ckf$ as the tautological section $C \to T_C$
associated to the tangent bundle of any open set in a vector space.
Since $\kf = \Im\kf$, we have $\d\kf = \frac{1}{2i} \d\ckf$ and 
$\dbar\kf = -\frac{1}{2i} \dbar\ckf$, where $d = \d + \dbar$ is
the exterior derivative on the vector space $\coho{1,1}(X,\CC)$.
We now have
\begin{equation*}
\displaylines{
-4 \cdot \tfrac{i}{2}\d\dbar \log \Vol
= 
\frac{i}{2}
\q{\d\ckf} \cdot \q{\dbar\ckf}
\hfill\cr\hfill{}
- \frac{i}{2} \qq{\d\ckf}{\dbar\ckf}
}
\end{equation*}
after being careful about sign errors. The Hermitian form associated to
this closed form is clearly $(1)$.
\end{proof}


We record here the basic commutation results on the Lefschetz operator
and its adjoint that we use.


\begin{prop}
Let $\kf$ be a K\"{a}hler class, $L$ the associated Lefschetz operator and
$\Lef$ its adjoint. The following hold for classes of degree $k$:
\begin{itemize}
\item $[L,\Lef] = (k-n)\id$.
\item $[L\^i,\Lef] = (k-n+i-1)L\^{i-1}$.
\item $[L,\Lef\^i] = (k-n+i-1)\Lef\^{i-1}$.
\end{itemize}
\end{prop}


\begin{proof}
The first identity is classical, the second is proved in
\cite[Corollary~1.64]{HuybrechtsGeometry}, and the third follows from
the second by taking adjoints. 
\end{proof}


\begin{rema}
The real version of this metric, defined on the K\"{a}hler cone, has been
studied by Wilson~\cite{Wilson} along with Trenner~\cite{WilsonTrenner},
mostly on the K\"{a}hler cone of manifolds with trivial canonical bundle.
Wilson picked a K\"{a}hler metric to work with and was able to express the
curvature tensor of the metric in terms of differential forms.

In my thesis~\cite{Magnusson} I also studied the real metric by
embedding the K\"{a}hler cone into the space of Hermitian metrics on the
underlying manifold via the Aubin--Calabi--Yau theorem (as in fact
suggested by Wilson in his paper). That space carries a natural
Riemannian metric, known as the Ebin metric when considered on the space
of Riemannian metrics on the underlying smooth manifold;
see~\cite{Ebin,ClarkeRubinstein}. This embedding lets us calculate the
curvature tensor of the metric.

The problem with both approaches is that while the metric on the K\"{a}hler
cone is defined entirely in cohomological terms, the expressions one
gets for its curvature tensor are not cohomological. They involve
differential forms that are induced from harmonic forms on the manifold,
but are themselves neither harmonic nor closed. This makes it difficult
to say anything at all about the curvature of the metric. In this paper
we'll treat the Aubin--Calabi--Yau theorem and actual K\"{a}hler metrics as
red herrings and stay wholly within the cohomology ring of our manifold.

One can also remark that Wilson chooses to work on a subset of the
K\"{a}hler cone; one defined by all K\"{a}hler metrics of a fixed volume. This
makes perfect sense in the real case, since the K\"{a}hler cone is isometric
to the positive real line times such a subset and this restriction lets
us work only with primitive forms. Our metric (or rather its real version)
is the same as Wilson's metric once restricted to this subset.  I don't
see a similar useful complex subspace of the complexified K\"{a}hler
cone so we work on all of it. This doesn't change anything important; if
anything it simplifies our life by letting us forget about projections
to primitive cohomology.
\end{rema}



\subsection*{The Chern connection and curvature form}

The Chern connection on a holomorphic vector bundle with a Hermitian
metric $h$ is the unique connection that is compatible with the metric
and whose $(0,1)$-part is $\dbar$; that is, it satisfies
$$
d h(\ton, \ov{\ttw}) 
= h(\chern \ton, \ov{\ttw}) + h(\ton, \ov{\chern \ttw}),
\quad
\chern^{0,1} = \dbar
$$
for all sections $u, v$ of the bundle.


\begin{prop}
\label{prop:chernconnection}
The Chern connection of the K\"{a}hler metric on $\KC$ is
$$
\chern \ton
= d\ton 
- \Lef(\d \kf) \, \ton
- \Lef(\ton) \, \d\kf 
+ \Lef(\ton \wedge \d\kf).
$$
\end{prop}


\begin{proof}
It's enough to calculate the Chern connection on local holomorphic
vector fields, so let $\ton, \ttw$ be thus. Then the Chern connection
satisfies
$$
\d h(\ton,\ov\ttw)
= h(\chern^{1,0}\ton, \ov\ttw),
$$
and knowing that is enough to describe it all (plus it saves us some
calculations). We have
$$
h(\ton,\ov\ttw)
= \q{\ton} \cdot \q{\ov\ttw} - \qq{\ton}{\ov\ttw}.
$$
First we note that
\begin{equation*}
\displaylines{
\d\q{\ton}  
= 
- \q{\d\kf} \cdot \q{\ton} 
\hfill\cr\hfill
{}+ \q{\d\ton}
{}+ \qq{\ton}{\d\kf}.
}
\end{equation*}
Second, we have
\begin{equation*}
\displaylines{
\d\qq{\ton}{\ov\ttw}
=
{}- \q{\d\kf} \cdot \qq{\ton}{\ov\ttw}
\hfill\cr\hfill
{}+ \qq{\d\ton}{\ov\ttw}
%\cr\hfill
{}+ \qqq{\ton}{\ov\ttw}{\d\kf}.
}
\end{equation*}
Putting these together, we take a deep breath and calculate
\begin{equation*}
\def\bil{\phantom{dh(\ton,\ttw)}}
\displaylines{
\d h(\ton,\ttw)
{}=
\d\biggl( \q{\ton} \cdot \q{\ov\ttw}\biggr)
- \d\biggl( \qq{\ton}{\ov\ttw}\biggr)
\hfill\cr
%\bil{}=
%\d\biggl( \q{\ton} \biggr) \cdot \q{\ov\ttw}
%\hfill\cr\hfill
%{}+ \q{\ton} \cdot \d\biggl( \q{\ov\ttw} \biggr) 
%\cr\hfill
%{}- \d\biggl( \qq{\ton}{\ov\ttw}\biggr)
%\cr
\bil
{}= 
\hfill
\textcolor{green}{- \q{\d\kf} \cdot \q{\ton} \cdot \q{\ov\ttw}}
%\quad(1)
\cr\hfill
\textcolor{red}{{}+ \q{\d\ton}\cdot \q{\ov\ttw}}
%\quad(2)
\cr\hfill
{}+ \qq{\ton}{\d\kf}\cdot \q{\ov\ttw}
%\quad(3)
\cr\hfill
\textcolor{blue}{{}- \q{\ton} \cdot \q{\d\kf} \cdot \q{\ov\ttw}}
%\quad(4)
\cr\hfill
\textcolor{blue}{{}+ \q{\ton} \cdot \qq{\ov\ttw}{\d\kf}}
%\quad(5)
\cr\hfill
\textcolor{green}{{}+ \q{\d\kf} \cdot \qq{\ton}{\ov\ttw}}
%\quad(6)
\cr\hfill
\textcolor{red}{{}- \qq{\d\ton}{\ov\ttw}}
%\quad(7)
\cr\hfill
{}- \qqq{\ton}{\d\kf}{\ov\ttw}.
%\quad(8)
\cr
\bil
{}=
\textcolor{red}{h(\d\ton, \ov\ttw)}
\textcolor{green}{- h(\d\kf, \kf) \cdot h(\ton, \ov\ttw)}
\textcolor{blue}{- h(\ton, \kf) \cdot h(\d\kf, \ov\ttw)}
\hfill
\cr\hfill
{}+ \qq{\ton}{\d\kf}\cdot \q{\ov\ttw}
\cr\hfill
{}- \qqq{\ton}{\d\kf}{\ov\ttw}.
}
\end{equation*}
By Lemma~\ref{lemm:hodgeproduct} there exists a unique vector $\ton
\star \d\kf \in \coho{1,1}(X,\CC)$ such that
\begin{align*}
\ton \star \d\kf \wedge \kf\^{n-2}
&= \ton \wedge \d\kf \wedge \kf\^{n-3},
\\
\tfrac{n-1}{n-2} \ton \star \d\kf \wedge \kf\^{n-1}
&= \ton \wedge \d\kf \wedge \kf\^{n-2}.
\end{align*}
Then we have
\begin{equation*}
\displaylines{
\frac{n-1}{n-2} 
\q{\ton \star \d\kf}\cdot\q{\ov\ttw}
- \qq{\ton \star \d\kf}{\ov\ttw}.
\hfill\cr\hfill{}
=\tfrac{1}{n-2} 
h(\ton \star \d\kf, \kf) \cdot h(\kf, \ov\ttw)
+ h(\ton \star \d\kf, \ov\ttw)
}
\end{equation*}
Putting this all together, we get
\begin{align*}
\chern \ton 
&= d\ton 
- h(\d \kf, \kf)\, \ton
- h (\ton, \kf)\, \d\kf 
+ \tfrac{1}{n-2} h(\ton \star \d\kf, \kf)\, \kf
+ \ton \star \d\kf
\\
&= d\ton 
- h(\d \kf, \kf)\, \ton 
- h (\ton, \kf)\, \d\kf 
+ \tfrac{1}{n-1} \Lambda\^{2}(\ton \wedge \d\kf)\, \kf
\\
&
\qquad \qquad \qquad \qquad
\qquad \qquad \qquad \qquad
+ \Lambda(\ton \wedge \tth) 
- \tfrac{1}{n-1} \Lambda\^{2} (\ton\wedge\tth)\, \kf
\\
&= d\ton 
- h(\d \kf, \kf)\, \ton 
- h (\ton, \kf)\, \d\kf 
+ \Lambda(\ton \wedge \d\kf),
\end{align*}
by using the expression of $\ton \star \d\kf$ from
Lemma~\ref{lemm:hodgeproduct}.
\end{proof}


\begin{lemm}
\label{lemm:hodgeproduct}
Let $\ton, \tth$ be $(1,1)$-classes on $X$. Then there is a unique
$(1,1)$-class $\ton \star \tth$ on $X$ such that
\begin{align*}
\ton \star \d\kf \wedge \kf\^{n-2}
&= \ton \wedge \d\kf \wedge \kf\^{n-3},
\\
\tfrac{n-1}{n-2} \ton \star \d\kf \wedge \kf\^{n-1}
&= \ton \wedge \d\kf \wedge \kf\^{n-2}.
\end{align*}
In fact,
\begin{equation*}
\ton \star \tth
= \Lambda(\ton \wedge \tth) 
- \tfrac{1}{n-1} \Lambda\^{2} (\ton\wedge\tth)\, \kf.
\end{equation*}
\end{lemm}


\begin{proof}
Let $L x = x \wedge \kf$ be the Lefschetz operator on $\coho{*}(X,\CC)$.
The hard Lefschetz theorem says that $L\^{n-2} : \coho{1,1}(X,\CC) \to
\coho{n-1,n-1}(X,\CC)$ is an isomorphism. Since $\ton \wedge \tth \wedge
\kf\^{n-3} = L\^{n-3}(\ton \wedge \tth)$ is an $(n-1,n-1)$-class, the
class $\ton \star \tth$ exists, is unique and equals
$(L\^{n-2})^{-1}(L\^{n-3}(\ton \wedge \tth))$.

Let $\Lambda$ be the adjoint of the Lefschetz operator. Recall that
$$
[L\^i, \Lambda] = (k - n + i - 1) L\^{i-1}
$$
on the space of $k$-classes on $X$; see
\cite[Corollary~1.64]{HuybrechtsGeometry}. On the space of $4$-classes
with $i = n-2$ this gives $[L\^{n-2}, \Lambda] = L\^{n-3}$, so
$$
L\^{n-3}(\ton \wedge \tth)
= L\^{n-2}\Lambda(\ton \wedge \tth) 
- \Lambda L\^{n-2}(\ton \wedge \tth).
$$
The first term is fine, so let's focus on the second one. We make the
banal remark that $\Lambda\^2(\ton \wedge \tth)$ is a scalar and
calculate in succession that
\begin{align*}
L\^{n-2}(\ton \wedge \tth) 
&= \Lambda\^{2}(\ton \wedge \tth)\, \kf\^n,
\\
\Lambda L\^{n-2}(\ton \wedge \tth) 
&= \Lambda\^{2}(\ton \wedge \tth)\, \kf\^{n-1},
\\
(\Lambda\^{n-2}a)^{-1}(\Lambda L\^{n-2}(\ton \wedge \tth) )
&= \tfrac{1}{n-1} \Lambda\^{2}(\ton \wedge \tth)\, \kf.
\end{align*}
This gives the first of our results. For the second we simply calculate
\begin{align*}
\ton \star \d\kf \wedge \kf\^{n-1}
= \frac{1}{n-1} \ton \star \d\kf \wedge \kf\^{n-2} \wedge \kf
&= \frac{1}{n-1} \ton \wedge \d\kf \wedge \kf\^{n-3} \wedge \kf
\\
&= \frac{n-2}{n-1} \ton \wedge \d\kf \wedge \kf\^{n-2},
\end{align*}
which is equivalent to the second result.
\end{proof}


\begin{lemm}
\label{coro:kahlerform}
$\chern^{1,0} \kf = -\d\kf$.
\end{lemm}


\begin{proof}
Let $\ton$ be a tangent field on $\KC$ and consider the equation
$\langle \ton, \kf \rangle = \Lef \ton$. Upon differentiating this in
the direction of $\ttw$ we get
$$
\langle \chern_{\ttw}\ton, \kf \rangle 
+ \langle \ton, \chern_{\ttw}\kf \rangle 
= (d_{\ttw}\Lef) \ton
+ \Lef d_{\ttw}\ton
= \tfrac{1}{2}i \langle \ton, \ttw\rangle
+ \Lef d_{\ttw}\ton.
$$
Proposition~\ref{prop:chernconnection} now gives
$$
\displaylines{
\langle \chern_{\ttw}\ton, \kf \rangle 
=
\Lef(d_{\ttw}\ton)
- i\Lef\ton\cdot \Lef\ttw
+ i\Lef\^2(\ton \wedge \ttw)
}
$$
so
\begin{align*}
\langle \ton, \chern_{\ttw}\kf \rangle
&=
-\tfrac{1}{2}i \Lef\ton \cdot \Lef \ttw
+\tfrac12 i \Lef\^2(\ton \wedge \ttw)
+ i\Lef\ton\cdot \Lef\ttw
- i\Lef\^2(\ton \wedge \ttw)
\\
&=
\langle \ton, -\tfrac{1}{2i} \ttw \rangle
= \langle \ton, -\d\kf(\ttw) \rangle.
\qedhere
\end{align*}
\end{proof}


\begin{theo}
The curvature tensor of the K\"{a}hler metric on $\KC$ is
\begin{equation*}
R(\ton,\ov\ttw,\tth,\ov\tfo)
= - \tfrac14 h(\ton,\ov\ttw)\, h(\tth,\ov\tfo)
- \tfrac14 h (\ton,\ov\tfo)\, h(\tth,\ov\ttw)
+ \tfrac14 \langle \ton \wedge \tth, \ov{\ttw\wedge\tfo} \rangle.
\end{equation*}
\end{theo}


\begin{proof}
Let $\ton$ be a local holomorphic section of the tangent bundle of $\KC$.
By definition, we have $\curv = \chern^2$ and since our connection is
Hermitian, we get $\curv = \dbar \chern$ on holomorphic sections.
Note that since $\kf = \Im \alpha$, we have $\d\dbar \kf = 0$.
Proposition~\ref{prop:chernconnection} and
Lemma~\ref{coro:kahlerform} now give
\begin{align*}
\dbar_{\ov\tfo} \chern_{\tth} \ton
&=
- \tfrac{1}{2i} \langle \tth,\ov{\chern^{1,0}_{\tfo}\kf} \rangle\, \ton
- \tfrac{1}{2i} \langle \ton, \ov{\chern^{1,0}_{\tfo}\kf} \rangle\, \tth
+ \tfrac{1}{2i} \dbar(\Lambda(\ton \wedge \tth))
\\
&=
- \tfrac{1}{4} \langle \tth,\ov{\tfo} \rangle\, \ton
- \tfrac{1}{4} \langle \ton, \ov{\tfo} \rangle\, \tth
+ \tfrac{1}{2i} \dbar(\Lambda(\ton \wedge \tth)).
\end{align*}
Upon taking the inner product with $\ttw$, we then get
$$
R(\ton,\ov\ttw,\tth,\ov\tfo)
= -\tfrac14
\langle \ton, \ov\ttw \rangle
\langle \tth,\ov{\tfo} \rangle
-\tfrac14
\langle \ton, \ov{\tfo} \rangle
\langle \tth, \ov\ttw \rangle
+ \tfrac{1}{4} \langle \ton \wedge \tth, \ov{\ttw \wedge \tfo}\rangle.
\qedhere
$$
\end{proof}


\begin{proof}[Alternate proof]
Let $\ton,\ttw,\tth,\tfo$ be local holomorphic sections of the tangent
bundle of $\KC$.  By definition, we have $\curv = \chern^2$ and since
our connection is the Chern connection, we get $\curv = \dbar \chern$ on
holomorphic sections. Proposition~\ref{prop:chernconnection} now gives
\begin{align*}
\dbar_{\ov\tfo} \chern_{\tth} \ton
&=
\dbar_{\ov\tfo} \d_{\tth}\ton
- \tfrac{1}{2i} ((\dbar_{\ov\tfo}\Lef) \tth)\ton
- \tfrac{1}{2i} ((\dbar_{\ov\tfo}\Lef) \ton)\tth
+ \tfrac{1}{2i} (\dbar_{\ov\tfo}\Lef)(\ton\wedge\tth)
\\
&=
- \tfrac{1}{4} \langle \tth,\ov{\tfo} \rangle\, \ton
- \tfrac{1}{4} \langle \ton, \ov{\tfo} \rangle\, \tth
+ \tfrac{1}{2i} (\dbar_{\ov\tfo}\Lef)(\ton \wedge \tth).
\end{align*}
Upon taking the inner product with $\ttw$, we then get
$$
R(\ton,\ov\ttw,\tth,\ov\tfo)
= -\tfrac14
\langle \ton, \ov\ttw \rangle
\langle \tth,\ov{\tfo} \rangle
-\tfrac14
\langle \ton, \ov{\tfo} \rangle
\langle \tth, \ov\ttw \rangle
+ \tfrac{1}{4} \langle \ton \wedge \tth, \ov{\ttw \wedge \tfo}\rangle.
\qedhere
$$
\end{proof}


\begin{coro}
The derived curvature tensors are as follows.

\smallskip
\noindent
$\bullet$
Holomorphic bisectional curvature:
$$
B(\ton,\ov\ttw) 
= 
\bigl(-\tfrac14 |\ton|^2 |\ttw|^2
- \tfrac14 |\met{\ton}{\ttw}|^2
+ \tfrac14 |\ton \wedge \ttw|^2
\bigr)
\bigm/
|\ton|^2|\ttw|^2
$$
\smallskip
\noindent
$\bullet$
Holomorphic sectional curvature:
$$
H(\ton) 
=
\bigl(
-\tfrac12 |\ton|^4
+\tfrac14 |\ton \wedge \ton|^2
\bigr)
\bigm/
|\ton|^4
$$
\smallskip
\noindent
$\bullet$
Ricci curvature:
$$
r(\ton, \ov\ttw) 
=
-\tfrac14(n+1) \met{\ton}{\ttw}
+ \sum_{j=1}^{h^{1,1}} \met{\ton\wedge x_j}{\ttw \wedge x_j},
$$
where $(x_1,\ldots,x_{h^{1,1}})$ is an orthonormal basis.
\end{coro}

\begin{coro}
The holomorphic sectional curvature of the metric is negative and
bounded; we have
$$
-\frac12 \leq H(\ton) \leq -\frac{1}{2n} < 0.
$$
The scalar curvature of the metric is also negative and bounded and
satisfies 
$$
-\frac{h^{1,1}(h^{1,1}+1)}{2} \leq s \leq -\frac{h^{1,1}(h^{1,1}+1)}{2n} < 0
$$
at all points on $\KC$.
\end{coro}


\begin{proof}
The lower bound on the holomorphic sectional curvature is immediate. For
the upper bound, C.N.~Yang proved that 
$$
\| \ton \wedge \ttw \|^2
\leq \frac{2n-2}{n} \|\ton\|^2 \|\ttw\|^2
$$
for $(1,1)$-forms on a complex vector space of dimension $n$.
(The constant is the best possible, as we see by taking multiples of the
Kahler form.) Now represent our classes by forms that are harmonic with
respect to an arbitrary Kahler metric in our Kahler class. Then this
inequality holds pointwise on the underlying manifold. The wedge product
of harmonic forms is usually not harmonic, so the integral of the
norm of the wedge product of our forms is not the cohomological norm; it
is bigger, because the wedge product of the harmonic forms represents the
wedge product of the classes, and the harmonic forms are the forms of
minimal volume in their equivalence class. Integrating over the
manifold, we thus obtain the inequality on the level of cohomology.
Applying this inequality to the holomorphic sectional curvature, we get
$$
H(\ton) 
\leq 
\tfrac14
\bigl( \tfrac{n-2}{n} \|\ton\|^4 - \|\ton\|^4\bigr) \bigm/ \|u\|^4
= -\tfrac{1}{2n}.
$$

For the scalar curvature, it is known that the average of the
holomorphic sectional curvature over the unit tangent sphere over a
point of our manifold is equal to $s / h^{1,1}(h^{1,1}+1)$. This gives
the announced result.
\end{proof}


\begin{rema}
The holomorphic bisectional curvature is also bounded below;
dropping the $|\ton\wedge\ttw|^2$ term and applying Cauchy--Schwarz
gives
$$
-\tfrac12 \leq B(\ton,\ttw).
$$
By constrast, Yang's estimate only gives
$$
B(\ton,\ttw)
\leq 
\tfrac14
\bigl( \tfrac{n-2}{n} \|\ton\|^2 \|\ttw\|^2 
- |\langle \ton, \ov\ttw\rangle|^2\bigr)
\bigm/ \|\ton\|^2\|\ttw\|^2,
$$
suggesting that the holomorphic bisectional curvature of this metric is
of mixed sign. 

Similarly, it's immediate that the Ricci curvature satisfies
$$
-\tfrac14(n+1)\|\ton\|^2 \leq r(\ton,\ov\ton),
$$
but Yang's estimate may not say anything about an upper bound; the naive
estimate involving the triangle inequality and Cauchy--Schwarz is too
brutal to give anything interesting.
\end{rema}




\subsection*{Completeness}

The theorem of Demailly and Paun describes the boundary of the
K\"{a}hler cone of a compact complex manifold. It consists of three
parts:
\begin{enumerate}
\item Limits of classes $a_t$ whose volume $\int_X a_t\^n$
tends to zero.
\item Limits of classes whose volume tends to infinity.
\item Limits of classes whose volume tends to some positive real
number, but there exists a proper irreducible complex subspace $Z
\subset X$ of dimension $p \geq 1$ whose volume tends to zero.
\end{enumerate}
Complexifying the K\"{a}hler cone adds one more entry to this list:
\begin{enumerate}
\item[(4)] Limits of classes that fall into none of cases (1)--(3), but
whose real part tends to infinity.
\end{enumerate}

Let us conspire to call $\mathcal{P} := \{\alpha \in \coho{1,1}(X,\CC)
\mid (\Im\alpha)^n > 0\}$ the cone of complexified volume classes on
$X$, or the complexified volume cone. It contains the complexified
K\"{a}hler cone, but is in almost all cases bigger than it.

\begin{prop}
\label{prop:fofo}
The metric on the complexified K\"{a}hler cone of $X$ is complete if and
only if cone is a connected component of the complexified volume cone.
\end{prop}

\begin{proof}
We first show that the classes on the first two parts of the boundary
pose no problems. Let $I$ be an interval in the real numbers and let
$\gamma : I \to \KC$ be a smooth path in $\KC$ that approaches the
boundary of $\KC$. Let $I_m = [a, b_m]$ be an increasing exhaustion
of $I$ by compact intervals and let $\gamma_m$ be the restriction
of $\gamma$ to $I_m$. Suppose that the volume $\Vol(X,\Im\gamma_m)$
tends to either zero or infinity as $m$ tends to infinity.

\begin{lemm}
Let $I = [a,b]$ be a compact interval in the real numbers $\RR$,
and let $\gamma : I \to \KC$ be a smooth path. The length of
the path $\gamma$  satisfies
$$
L(\gamma) \geq
\frac{\sqrt 2}{\sqrt n}
\bigl| \log \Vol(X,\Im\gamma(b))
- \log \Vol(X,\Im\gamma(a))
\bigr|.
$$
\end{lemm}

\begin{proof}[Sketch of proof.]
We apply the Cauchy--Schwarz inequality to the scalar product
$h(\ton,\kf)$; this gives
$$
|\ton \cdot \log \Vol(X,\kf)|^2 
= |\tfrac{1}{2i}h(\ton,\kf)|^2 \leq \tfrac{n}{2} h(\ton,\ov\ton).
$$
Integrating and applying the triangle inequality then gives the
announced estimate.
\end{proof}

Applying the lemma on each interval $I_m$ then gives that
\begin{equation*}
  L(\gamma) = \lim\limits_{m \to +\infty} L(\gamma_m) = +\infty.
\end{equation*}
Thus the limit class $\lim \gamma(t)$ on the boundary cannot be
approached by paths in $\KC$ of finite length.

If the complexified K\"{a}hler and volume cones of $X$ do not
coincide, then there exists a class $\alpha$ on the boundary of
$\KC_{\RR}$ such that $\Vol(X,\alpha) > 0$, but there is a proper
complex subspace $Z \subset X$ such that $\Vol(Z,\alpha) = 0$.

As $\alpha$ is on the boundary of the K\"{a}hler cone, then there
exists a K\"{a}hler class $\kf$ such that $\gamma(t) := \alpha +
t\kf$ is in the K\"{a}hler cone for all $t > 0$. The tangent vectors
of the path $\gamma$ are $\gamma'(t) = \kf$, and the norm of
$\gamma'(t)$ at the point $\gamma(t)$ is
$$
\displaylines{
  h(t) :=
  g(\gamma'(t), \gamma'(t))(\gamma(t)) =
  \left(
    \frac{1}{\Vol(X,\gamma(t))}
    \int_X \kf \wedge (\alpha + t\kf)\^{n-1}
  \right)^2
\hfill\cr\hfill
{}- \frac{1}{\Vol(X,\gamma(t))}
    \int_X \kf^2 \wedge (\alpha + t\kf)\^{n-2}.
}
$$
Each of these integrals, and the function $t \mapsto \Vol(X,\gamma(t))$,
is a polynomial in $t$ on some small interval $[0,t_0]$. As $\lim_{t\to
0} \Vol(X,\gamma(t)) > 0$ the function $t \mapsto h(t)$ is continuous
and positive on a compact interval, so the integral $L(\gamma)$ of its
square root exists and is finite.

Finally suppose that we have a path $\gamma : I = [a,b] \to C$ that
doesn't fit into cases (1)--(3) but meanders along the real direction in
the complexified K\"{a}hler cone. I then claim that the set
$$
J := \{ \Im \gamma(t) \mid t \in I \}
$$
is relatively compact in $C_{\RR}$: Let $\nu$ be a limit point of that
set. Since the volumes of the classes $\Im \gamma(t)$ neither tend
towards $0$ nor $\infty$ they are contained in some compact interval, so
the volume of $\nu$ is positive. Similarly, we've excluded that the
volume of some subvariety $Z$ with respect to $\nu$ is zero by
hypothesis, so $\Vol(Z,\nu) > 0$ for all subspaces $Z$ of $X$. Thus
$\nu$ is a K\"{a}hler class. 
This implies that
$$
%m(t) := \inf_{\nu \in J} h(\gamma'(t),\ov{\gamma'(t)})_{\nu}
%\quad\text{and}\quad
M(t) := \sup_{\nu \in J} h(\gamma'(t),\ov{\gamma'(t)})_{\nu}
$$
exists, so the length of $\gamma$ can be bounded by above by
$\int_I M(t) dt$.
\end{proof}

\subsection*{Pullbacks}

A holomorphic map $f : X \to Y$ between compact K\"{a}hler manifolds induces
a morphism $f^* : \coho{*}(Y,\CC) \to \coho{*}(X,\CC)$ in cohomology
that respects the Hodge decomposition. However, if $\kf$ is a K\"{a}hler
class on $Y$, then $f^*\kf$ is hardly ever a K\"{a}hler class on $X$. This
happens mostly if $f$ is either an embedding or a finite covering map.

\begin{prop}
Let $f : X \to Y$ be a finite surjective morphism. Let $h_X$ and $h_Y$
be the K\"{a}hler metrics on the complexified K\"{a}hler cones of $X$ and
$Y$, respectively. Then the pullback morphism $f^* : \KC(Y) \to \KC(X)$
is a Hermitian embedding.
\end{prop}

\begin{proof}
Let $\ckf$ be a point in $\KC(Y)$ and $\kf = \Im \ckf$. The volume of
$X$ with respect to $f^*\kf$ is
\begin{equation*}
  \Vol(X,f^*\kf) = p \, \Vol(Y,\kf)
\end{equation*}
as $f$ is finite of degree $p$. It follows that $f^*$ is an embedding.
\end{proof}

\begin{coro}
The group $\Aut X$ of holomorphic automorphisms of $X$ acts by
isometries on the K\"{a}hler cone $\KC(X)$.
\end{coro}

A closer look reveals that this last statement contains less information
than first meets the eye. The automorphism group $\Aut X$ of a compact
complex manifold is a Lie group and it splits roughly into two parts; a
positive-dimensional group given by the flows of holomorphic vector
fields, or elements of $\coho{0}(X,T_X)$, and a discrete part consisting of
``other'' automorphisms. The isomorphisms generated by vector fields act
trivially on the cohomology ring of $X$, so the only part of $\Aut X$
that possibly acts by nontrivial isometries on $\KC(X)$ is discrete.


\section{The relative K\"{a}hler cone}
\label{sefi}


\subsection*{Complex structure and connection}

Let $\pi : \cX \to S$ be a family of compact K\"{a}hler manifolds over
a smooth base $S$. Recall that there is a holomorphic vector bundle
$E^{1,1} \to S$ whose fibers are $E^{1,1}_s = H^{1,1}(X_s,\CC)$. The
\emph{complexified relative K\"{a}hler cone} of a family $\pi : \cX
\to   S$ is the subset $\RKC$ of $p : E^{1,1} \to S$ that consists
of the complexified K\"{a}hler cones of each manifold $X_s$.

\begin{prop}
The relative K\"{a}hler cone $\RKC$ is open in the total space of
the vector bundle $E^{1,1}$.
\end{prop}

\begin{proof}[Sketch of proof.]
We adapt the proof of Kodaira--Spencer \cite{KodairaSpencerIII} of
the fact that the K\"{a}hler condition is open in families. Given a
point $(a_0,s_0)$ in $\RKC$, we find a relative K\"{a}hler metric that
interpolates that point. The metrics thus obtained on each manifold
in the family permit us to identify cohomology classes with harmonic
forms on each manifold.

After restricting to the inverse image of a relatively compact
neighborhood of the point $s_0$, we note that the unit ball
fibration in $T_{\cX/S}$ is compact over the closure of that
neighborhood. Since positivity of forms can be tested on that
fibration, we obtain an open ball in $\RKC$ around $(a_0,s_0)$
that is contained in $E^{1,1}$.
\end{proof}

The proposition entails that the complexified relative K\"{a}hler cone is
a complex manifold. It is equipped with a surjective submersion $p :
\RKC \to S$ inhereted from its ambient vector bundle, but is not
necessarily locally trivial since the K\"{a}hler cone may vary in
families
\cite{DemaillyPaun}.


The vector bundle $E^{1,1} \to S$ is equipped with a smooth connection
$\conn$; it is the connection induced by embedding $E^{1,1}
\hookrightarrow E^2$ smoothly, applying the Gauss--Manin connection on
$E^2$ and projecting the result onto $(1,1)$-classes. Since $\RKC$
is open in $E^{1,1}$, it inherits this connection.


\begin{prop}
The $(0,1)$-part of $\conn$ is $\dbar$.
\end{prop}


\begin{proof}
If $u$ is a local section of $E^{1,1}$, then by definition we have
$$
\conn^{0,1} u = q (\conngm^{0,1} j(u))
= q ((\dbar j) (u))
+ q (j(\dbar u))
= q ((\dbar j) (u))
+ \dbar u. 
$$
If the inclusion morphism $j : E^{1,1} \hookrightarrow E^2$ were
holomorphic (which happens if the two bundles are equal) then we'd have
$\dbar j = 0$ and be done.  Since in general it isn't we have to
work a little more.

Consider the Grassmannian $G := \Gr(h^{1,1}, H^2(X_{s_0},\CC))$ of
$h^{1,1}$-dimensional complex subspaces of $H^2(X_{s_0},\CC)$, where
$s_0$ is some point in the neighborhood our section $u$ is defined on.
The inclusion map $j : E^{1,1} \hookrightarrow E^2$ can be viewed as a
smooth map $j : S \to G$ and the Hodge decomposition gives an orthogonal
complement $j(s)^{\perp} = H^{2,0}(X_s, \CC) \oplus H^{0,2}(X_s, \CC)$ to
$j(s) = H^{1,1}(X_s, \CC)$ for every point $s$. By describing an open
neighborhood in the Grassmannian around $j(s)$ using this orthogonal
splitting, we see that $\dbar j$ corresponds to
$$
\dbar j : T_{S,s}^{0,1}
\to 
T_{G, j(s)}^{0,1}
\cong
\Hom_{\ov{\CC}}(j(s), j(s)^{\perp}).
$$
But since $q$ is the orthogonal projection onto $j(s)$, we have
$q(\dbar j) = 0$.
\end{proof}


\subsection*{Closed Hermitian form}

Consider the relative tangent space $T_{\RKC/S}$ over $\RKC$. It carries
a natural smooth Hermitian metric, just as in
Section~\ref{section:fixemanifold}, that can be defined in any of the
same ways.

\begin{theo}
The Hermitian metric on $T_{\RKC/S}$ extends to a closed semipositive
Hermitian form $\kcForm$ on all of $\RKC$.
\end{theo}


\begin{proof}
We define $\kcForm = -4 \frac{i}{2}\partial\dbar \Vol$, where
$\Vol : \RKC \to \RR$ is the volume function. Since $\Vol$ is
smooth and real, this defines a smooth, real $(1,1)$-form on
$\RKC$, whose associated Hermitian form $\kcHerm$ on $T_{\RKC}$
satisfies $\kcHerm(X, Y) = -4\partial_{X} \dbar_{\ov{Y}} \Vol$ and
obviously extends the metric on the relative tangent space to the whole
tangent space.

The form is closed by construction, so it remains to prove that it is
semipositive on $\RKC$.  The associated Hermitian form $\kcHerm$ is
positive-definite on $T_{\RKC/S}$, so it induces a smooth splitting
$T_{\RKC} \cong T_{\RKC/S} \oplus p^*T_S$. 

\begin{lemm}
\label{lemm:orthogonal.split}
In a local holomorphic trivialization $E^{1,1}|_U \cong \CC^r \times U$,
the splitting defined by $\kcHerm$ on $T_{\RKC}$ is 
$$
X 
= X^{\mathrm{V}} + X^{\mathrm{H}}
= (\ton + 2i \conn_{\xi}\kf)
+ (\xi - 2i \conn_{\xi}\kf),
$$
where $X = \ton + \xi$ under the trivialization.
\end{lemm}

Assume the lemma for a moment. Then we calculate that
\begin{align*}
X^{\mathrm{H}} \cdot \Vol
&= p_*X \cdot \int_{X_s} \kf\^n
- 2i\conn_{p_*X}\kf \cdot \int_{X_s} \kf\^n
\\
&= \int_{X_s} \conn_{p_*X}\kf \wedge \kf\^{n-1}
- \int_{X_s} \frac{1}{2i} 2i \conn_{p_*X}\kf \wedge \kf\^{n-1}
= 0,
\end{align*}
so the curvature form degenerates on the whole of the horizontal part of
the smooth splitting. That only leaves the vertical part, where we
already know it is positive-definite.
\end{proof}


\begin{proof}[Proof of Lemma~\ref{lemm:orthogonal.split}]
The orthogonal splitting is described by the projection $T_{\RKC} \to
T_{\RKC/S}$ that sends a tangent vector $X$ of $\RKC$ to the unique
section $X^{\mathrm{V}}$ of $T_{\RKC/S}$ that satisfies
\begin{equation}
\label{eq:smooth.split}
\kcHerm(X, \ov{\ttw}) = \kcHerm(X^{\mathrm{V}}, \ov{\ttw})
\end{equation}
for all local sections $\ttw$ of $T_{\RKC/S}$.

Let's take a local trivialization $E^{1,1}|_U \cong \CC^r \times U$
and identify the relative complexified K\"{a}hler cone with some open
set in $\CC^r \times U$. Then the tangent bundle of $\RKC$ splits
holomorphically as $T_{\RKC/S} \oplus p^* T_S$ over $p^{-1}(U)$.

First we note that if $\xi$ is a local section of $p^*T_S$ and $v$ a
local holomorphic section of $T_{\RKC/S}$, then
\begin{align*}
\kcHerm(\xi, \ov{\ttw})
&=
-4\partial_{\xi} \dbar_{\ov{\ttw}} \log \Vol
\\
&=
-4\partial_{\xi}
\biggl(
\frac{1}{\Vol} 
\int_{X_s} \frac{-1}{2i} \ov{\ttw} \wedge \kf\^{n-1}
\biggr)
\\
&= 
-4
\biggl(
\frac{1}{\Vol} 
\int_{X_s} -\conn_{\xi} \kf \wedge \kf\^{n-1}
\cdot
\frac{1}{\Vol} 
\int_{X_s} \frac{-1}{2i} \ov{\ttw} \wedge \kf\^{n-1}
\\
&
\qquad
\quad
- \frac{1}{\Vol}
\int_{X_s} 
\biggl(
\frac{-1}{2i} \conn_{\xi}\ov{\ttw} \wedge \kf\^{n-1}
+
\frac{-1}{2i} \conn_{\xi}\kf \wedge \ov{\ttw} \wedge \kf\^{n-2}
\biggr)
\biggr)
\\
&= 
-4
\biggl(
\frac{1}{\Vol} 
\int_{X_s} \frac{1}{2i}\conn_{\xi} \kf \wedge \kf\^{n-1}
\cdot
\frac{1}{\Vol} 
\int_{X_s} \ov{\ttw} \wedge \kf\^{n-1}
\\
&
\qquad
\qquad
\qquad
\qquad
\qquad
\qquad
\qquad
-
\frac{1}{\Vol}
\int_{X_s} 
\frac{1}{2i}\conn_{\xi}\kf \wedge \ov{\ttw} \wedge \kf\^{n-2}
\biggr)
\\
&=
\kcHerm(2i\conn_{\xi}\kf, \ov{\ttw}),
\end{align*}
where $\ov{\conn_{\xi}\ov{\ttw}} = \conn_{\ov{\xi}}\ttw = 0$ because
the $(0,1)$-part of $\conn$ is $\dbar$.  Let now $X = \ton +
\xi$ be a tangent field on $\RKC$. Then we have 
$$
\kcForm(X, \ov{\ttw})
= \kcForm(\ton + 2i \conn_{\xi}\kf, \ov{\ttw})
$$
for all $v$, so $X^{\mathrm{V}} = \ton + 2i \conn_{\xi}\kf$. It
follows that the other part of the splitting is $X^{\mathrm{H}} = X
- X^{\mathrm{V}} = \xi - 2i \conn_{\xi}\kf$.
\end{proof}


\begin{coro}
\label{coro:kahlermanifold}
If the base $S$ is K\"{a}hler, then $\RKC$ is a K\"{a}hler manifold.
\end{coro}


\begin{coro}
The semipositive form $\kcForm$ is a constant multiple of the curvature
form of a holomorphic line bundle.
\end{coro}

\begin{proof}
Consider the holomorphic line bundle $E^{2n}$ associated to the
coherent sheaf $\cR^{2n} \pi_* \CC \otimes_\CC \cO_S$ over $S$.
We pull this line bundle back to $\RKC$ and equip it with a smooth
Hermitian metric $g$ induced by the K\"{a}hler classes associated to each
point of $\RKC$. Our claim is that the curvature form of this
metric is $-1/4\pi$ times our form. To verify this, we describe the
holomorphic structure of $E^{2n}$ and the metric in more detail.

First we claim that the section
$$
\tau: \RKC \to p^*E^{2n},
\quad
(\ckf,s) \mapsto \kf\^n / \Vol(X,\kf),
$$
where $\kf = \Im \ckf$, is holomorphic and trivializes the line
bundle $p^*E^{2n}$. For this, note that the section $\tau$ is constant
on the fibers of $p : \RKC \to S$.  Next remark that $\tau$ satisfies
\begin{equation*}
\int_{X_s} \tau(\ckf,s) 
= \frac{1}{\Vol(X_s,\kf)} \int_{X_s} \kf\^n
= 1
\end{equation*}
at all points $(\ckf,s)$ of $\RKC$. It is thus dual to the
fundamental class of each manifold $X_s$, so it is parallel with
respect to the pullback of the Gauss--Manin connection on $E^{2n}$ to
$\RKC$, and thus holomorphic. The section $\tau$ is also clearly
nowhere zero, and thus trivializes $p^*E^{2n}$.

General complex differential geometry now tells us that the curvature
form of the metric on $p^*E^{2n}$ is
\begin{equation*}
\curv_{p^*E^{2n}, g}
= -\tfrac{i}{2\pi}\partial \dbar \log |\tau|^2 
= \tfrac{i}{2\pi}\partial \dbar \log \Vol(X_s,\kf)
\end{equation*}
at a point $(\ckf,s)$, and this is $-1/4\pi$ times our form.
\end{proof}



We've extended the Hermitian metric on $T_{\RKC/S}$ to all of
$T_{\RKC}$ by pulling a global potential out of our hat. Since we
have a connection $\conn$ on $\RKC$, we could also have used the
smooth splitting provided by the connection to split the tangent bundle
and extend our Hermitian metric to the whole bundle by setting it to
zero on the horizontal part of the splitting. This does in fact result
in the same metric on all of $\RKC$ because of the following:


\begin{prop}
The smooth splitting of $T_{\RKC}$ defined by $\conn$ is the same
as the splitting defined by $\kcForm$.
\end{prop}


\begin{proof}
Let $X$ be a tangent field on $\RKC$ near a point
$(\ckf,s)$ and let $\tau : S \to \RKC$ be a section that is
parallel at $(\ckf,s)$. The projection onto the vertical subspace
defined by $\conn$ is then $X \mapsto X - d_{p_*X} \tau$. We note that
$$
\kcHerm(X - d_{p_*X} \tau, \ov{\ttw})
= \kcHerm(X, \ttw) - \kcHerm(d_{p_*X} \tau, \ttw)
$$
for all local sections $v$ of $T_{\RKC}$. The smooth inclusion
$j : \RKC \hookrightarrow E^2$ lets us consider $\tau$ as a smooth
section of $E^2$. This bundle is flat, so the Gauss--Manin connection
identifies with the exterior derivative $d$. Now, $\tau$ is parallel
with respect to $\conn$ if and only if the class $\conngm j(\tau) = d
(j(\tau))$ decomposes into $(2,0)$ and $(0,2)$-classes only. Then the
intersection product
$$
\kcHerm(d_{p_*X} \tau, \ttw) = 0
$$
because the $(1,1)$-class $\ttw$ is orthogonal to $(2,0)$ and
$(0,2)$-classes. Thus $\kcHerm(X - d_{p_*X} \tau, \ov{\ttw})
= \kcHerm(X, \ttw)$, and the splitting defined by $\conn$ is the same
as the one defined by $\kcHerm$.
\end{proof}


\subsection*{Curvature}


Calculate the curvature form of $T_{\RKC/S}$.


Suppose we're given a Hermitian metric $h_S$ on the base $S$. Then we
can equip the relative complexified Kahler cone with $h = \kcHerm + 
p^*h_S$, which is then a smooth Hermitian metric and Kahler if $h_S$ is
Kahler. We can calculate the curvature tensor of this metric if we know
the second fundamental form our splitting induces.




\section{Zero first Chern class}


We now restrict our attention to compact Kahler manifolds with zero
first Chern class. The class of such manifolds includes complex tori, K3
surfaces, Calabi--Yau manifolds and hyperk\"{a}hler manifolds. They have
a rich structure that permits us to say more about them than about
arbitrary Kahler manifolds.


\subsection*{Weil--Petersson metric}


Let $\pi : X \to S$ be a family of compact K\"ahler manifolds of
dimension $n$ with zero first Chern class over a smooth base $S$. If
the family is polarized, that is equipped with a smoothly varying
family of K\"ahler classes whose cohomology class is constant in $s$,
then we know how to construct a Weil--Petersson metric on the base $S$;
see \cite{MR829406,MR784145,Tian,wang2003curvature}.

This is unsatisfactory because not every family of such manifolds can be
polarized. It also begs the question of whether two different
polarizations result in the same Weil--Petersson metric. By working over
the relative complexified Kahler cone we can avoid choosing a
polarization, and still construct a satisfying Hermitian metric that
gives rise to the Weil--Petersson one when the family can be polarized.



To begin with, let $\pi : X \to S$ be a family of compact K\"ahler
manifolds of dimension $n$ with trivial canonical bundle over a smooth
base $S$.
Consider the variation of Hodge structures $\rel \subset \CR^n\pi_*\CC
\otimes_\CC \CO_S$. The intersection product on each manifold $X_s$
(multiplied by $i^{n^2}$) defines a pseudo\-hermitian metric
on the flat holomorphic vector bundle $\CR^n\pi_*\CC \otimes_\CC
\nobreak\CO_S$, and the Gauss--Manin connection is the Chern connection
of this metric. Its restriction to the holomorphic sub-line bundle
$\rel$ is positive-definite by the Hodge--Riemann bilinear relations
and thus defines a hermitian metric $h$ on that bundle.

\begin{theo}
  \label{theoon}
  The hermitian line bundle $(\rel, h) \to S$ is seminegative,
  and negative if the family $\pi : X \to S$ is effective. If the
  family can be polarized, then the Weil--Petersson metric given by
  the polarizaton is a constant multiple of the curvature form of $h$.
\end{theo}


The next two propositions contain calculations necessary to our
efforts. We try to keep our exposition brief, perhaps at the expense
of clarity, but all the details are in \cite{Magnusson}.


\begin{prop}
  \label{propon}
  Let $\omega$ be a Ricci-flat K\"ahler metric on $X$. If $\sigma$
  is a nowhere zero holomorphic $(n,0)$-form on $X$ such that
$$
  \int_X \frac{i^{n^2}}{2^n} \sigma \wedge \overline \sigma
  = \Vol(X,\omega),
$$
then the morphism
$$
  H^1(X,T_X) \to H^{n-1,1}(X,\CC),
  \quad u \mapsto u \cup \sigma,
$$
is an isometry of cohomology groups,
where the groups are equipped with the $L^2$ inner products defined
by $\omega$.
\end{prop}

\begin{proof}
  Since $\omega$ is Ricci-flat then $\omega^n/n! = (i^{n^2}/2^n)
  \sigma \wedge \overline{\sigma}$ under our hypothesis. Now represent
  all cohomology groups in sight by $\omega$-harmonic forms, some
  with values in $T_X$. Fix a point $x \in X$. The result holds
  for the local morphism $\overline{T}^*_{X,x} \otimes T_{X,x} \to
  \bigwedge^{n-1} T_{X,x} \otimes \overline T_{X,x}^*$, which can be
  seen by picking an orthonormal basis of $T_{X,x}$ and calculating. It
  follows that the cup product is an isometry on the space of smooth
  sections equipped with the $L^2$ norm.

Since $\sigma$ is closed, the cup product sends closed forms to closed
forms. As harmonic representatives are the ones of minimal $L^2$ norm
in their cohomology class, the cup product then preserves harmonic
representatives, and is an isometry on the level of cohomology groups.
\end{proof}



\begin{prop}
\label{proptw}
Let $\pi : X \to S$ be a family of compact manifolds with trivial
canonical bundle over a smooth base $S$. Let $\xi$ and $\eta$ be
local holomorphic vector fields on $S$. Let $\sigma$ be a local
holomorphic nowhere zero section of $\rel$ over $S$, and let $\rho$
be the Kodaira-Spencer morphism of the family. Then
$$
\frac{i}{2\pi} \Theta_{\rel,h}(\xi, \overline \eta) =
\frac{i^{n^2}}{\| \sigma \|_{h}^2}
\int_{X_s} \rho(\xi) \cup \sigma
\wedge \overline{\rho(\eta) \cup \sigma}.
$$
\end{prop}

\begin{proof}
  We have
$$
\| \sigma \|_{h}^2 = \int_{X_s} i^{n^2} \sigma \wedge \overline \sigma.
$$
As usual, $i\Theta_{\rel,h}(\xi,\overline \eta)$ can be found by
differentiating the logarithm of this expression in the directions
of $\xi$ and $\overline \eta$. This maneuver will involve integrals
of the Gauss--Manin connection applied to $\sigma$.

  Griffiths transversality gives $\nabla_\xi \sigma = \nabla_\xi
  \sigma^{(n,0)} + \nabla_\xi \sigma^{(n-1,1)}$, and basic
  techniques in variation of Hodge structures show that $\nabla_\xi
  \sigma^{(n-1,1)} = \rho(\xi) \cup \sigma$. Consequently
$$
  \nabla_\xi \sigma =
  f_\xi \, \sigma + \rho(\xi) \cup \sigma,
$$
where $f_\xi$ is a complex function on $S$. A similar decomposition
holds for $\nabla_\eta \sigma$. Then purely formal calculations give the
announced result after a few lines.
\end{proof}


\begin{lemm}
\label{lemm:inner.product.top.degree}
Let $\kf$ be a Kahler class on $X$ and let $\alpha,\beta$ be
$(n-1,1)$-classes. Then
$$
\llangle \alpha, \ov\beta \rrangle
= -i^{n^2} \int_X \alpha \wedge \ov\beta.
$$
\end{lemm}


\begin{proof}
Decompose the given forms as
\begin{equation*}
\alpha = \alpha_0 \wedge \omega + \alpha_1, 
\quad
\beta = \beta_0 \wedge \omega + \beta_1,
\end{equation*}
where the forms $\alpha_0$ and $\beta_0$ are primitive $(n-2,0)$-forms
on $X$, and the forms $\alpha_1$ and $\beta_1$ are primitive
$(n-1,1)$-forms.  We observe that
\begin{align*}
  \llangle \alpha, \overline \beta \rrangle
  &= \llangle \alpha_0 \wedge \omega,
  \overline \beta_0 \wedge \omega \rrangle
  + \llangle \alpha_1, \overline \beta_1 \rrangle,
  \\
i^{n^2} \int_X \alpha \wedge \overline \beta
  &= 2i^{n^2}\int_X \alpha_0 \wedge \overline \beta_0 \wedge \omega\^2
  + i^{n^2}\int_X \alpha_1 \wedge \overline \beta_1
  \\
&= 2 \llangle \alpha_0, \beta_0 \rrangle
- \llangle \alpha_1, \beta_1 \rrangle
\end{align*}
by orthogonality of different primitive classes, degree reasons and the
hard Lefschetz theorem (see~\cite[Chapter 1.2]{HuybrechtsGeometry}). 
We note that $[L,\Lef] = -2\id$ and $\Lef = 0$ on $(n-2,0)$-classes.
Then
$$
\llangle \alpha_0 \wedge \omega,
\overline \beta_0 \wedge \omega \rrangle
= 2 \llangle \alpha_0, \overline \beta_0 \rrangle,
$$
so
$$
i^{n^2} \int_X \alpha \wedge \overline \beta
= \llangle \alpha_0, \overline \beta_0 \rrangle
- \llangle \alpha_1, \overline \beta_1 \rrangle.
\qedhere
$$
\end{proof}


\begin{proof}[Proof of Theorem~\ref{theoon}]
Let $s$ be a point of $S$. If we fix a Ricci-flat K\"ahler metric
$\omega$ on $X_s$ that satisfies their hypotheses, then
Propositions~\ref{propon} and \ref{proptw} and
Lemma~\ref{lemm:inner.product.top.degree} show that 
$$
  \frac{i}{2\pi} \Theta_{\rel,h}(\xi, \overline \xi) =
  -\frac{2^n}{\Vol(X_s,\omega)} \| \rho(\xi) \|^2_\omega \leq 0,
$$
where $\rho : T_{S,s} \to H^1(X_s,T_{X_s})$ is the Kodaira--Spencer
morphism of the family at $s$. If the family is effective, then $\rho$
is everywhere injective, so $(\rel,h)$ is negative.

Suppose now that the family is polarized by a family $\omega_s$ of
Ricci-flat K\"ahler metrics. Then $\nabla \omega_s = 0$, so $s \mapsto
\Vol(X_s,\omega_s) =: v$ is a constant function on $S$. If we apply
the above propositions again, or note if the norms of inner products
agree then they are the same by polarization identities, we see that
$$
\displaylines{
\hfill
\frac{i}{2\pi} \Theta_{\rel,h}(\xi, \overline \eta)
= -\frac{2^n}{v} \langle\!\langle \xi, \overline\eta
\rangle\!\rangle_{\mathrm{WP}}.
\hfill\qedhere
}
$$
\end{proof}

Wang \cite{wang2003curvature} calculated the curvature tensor
of the polarized Weil--Petersson metric from this Hodge-theoretic
viewpoint. We invite the reader to check that his calculations never
actually use the fact that the family in question is polarized,
so Wang really proves this:

\begin{theo}
  Let $\pi : X \to S$ be an effective family of compact K\"ahler
  manifolds with trivial canonical bundle. Let $\sigma$ be a local
  nowhere zero section of $\rel$. The curvature tensor of the K\"ahler
  metric $g$ defined by the curvature form of $({\rel}^*,-h)$ is
$$
\displaylines{
    R(\xi,\overline \eta, \nu, \overline \zeta) =
    -\bigl(g(\xi,\overline \eta)
    \, g (\nu,\overline \zeta)
    + g(\xi,\overline \zeta)
    \, g (\eta,\overline \nu) \bigr)
    \hfill\cr\hfill
    {}+ \frac{i^{n^2}}{\|\sigma\|^2_h}
    \int_{X_s} \rho(\xi) \cup \rho(\nu) \cup \sigma
    \wedge \overline{ \rho(\eta) \cup \rho(\zeta) \cup \sigma},
}
$$
where $\rho$ is the Kodaira-Spencer morphism of $\pi : X \to S$.
\end{theo}

As a consequence the usual negativity results on the curvature of the
Weil--Petersson metric carry over to $g$. For example, the holomorphic
sectional and Ricci curvatures of $g$ satisfy the bounds
$H \geq -2$ and $r \geq -(n+1)$.


\begin{exam}
  Let $S$ be the set of $n \times n$ complex matrices $s$ such that
$$
\det \Im s = \det \tfrac 1{2i}(s - \overline s) \not= 0.
$$
Each such a matrix defines a complex torus $X_s = \CC^n / (\ZZ^n \oplus
s \ZZ^n)$ of complex dimension $n$. The associated family $\pi : X
\to S$ is the universal family of complex tori $X_s$ that are marked
with a choice of basis of $H_1(X_s, \ZZ)$. Recall that principally
polarized Abelian varieties are parametrized by the subvariety
$$
A = \{ s \in S \mid {}^t s = s, \, \Im s \text{ is positive-definite}
\}.
$$

The family $\pi : X \to S$ cannot be polarized if $n > 1$; it is easy
to show that given a $(1,1)$-class $u \not= 0$ on some $X_s$, there is
always an infinitesimal direction $\xi$ of deformation such that $u$
is not of type $(1,1)$ on tori in the direction of $\xi$. Our results
nevertheless show that there is a Weil--Petersson metric $g$ on $S$,
and that the standard Weil--Petersson metric on the subvariety of
Abeliean varieties is the restriction of $g$ to $A$. In fact, if $\xi,
\eta \in T_{S,s} = M_n(\CC)$, then
$$
g(\xi,\overline\eta)
= \operatorname{tr}(\xi \cdot (\Im s)^{-1} \cdot
{}^t\overline \eta \cdot {}^t\!(\Im s)^{-1}),
$$
which clearly restricts to the usual Weil--Petersson metric on
the subvariety $A$ where the matrices $s$ are symmetric; see
\cite{MR784145}.
\end{exam}



Recall that if $X_0$ is a compact K\"ahler manifold with zero first
Chern class, then there exists an unramified finite covering
$\widetilde{X}_0 \to X_0$ such that $\widetilde{X}_0$ is compact and has
trivial canonical bundle; see \cite{MR730926}. Finite coverings can be
taken in families, so we can reduce the general case to the one of
manifolds with trivial canonical bundle. I don't have a reference for
this fact, so we provide a statement and proof sketch.



\begin{prop}
Let $\pi : X \to S$ be a family of compact complex manifolds with
finite fundamental group. Then there exists a family $\widetilde \pi :
\widetilde X \to S$ of simply connected compact complex manifolds and a
finite morphism of families $f : \widetilde X \to X$ such that
$\widetilde X_s$ is the universal cover of $X_s$ and $f_s : \widetilde
X_s \to X_s$ is the covering map for all $s \in S$.
\end{prop}

\begin{proof}
Let $f_0 : \widetilde X_0 \to X_0$ be an unramified holomorphic finite
covering map, where $\widetilde X_0$ is a compact complex manifold.
Let's write $X_0 = M$ and $\widetilde X_0 = \widetilde M$ for the
underlying smooth manifolds of $X_0$ and $\widetilde X_0$. Then $f_0 :
\widetilde M \to M$ is a smooth finite covering map. Given a complex
structure $J_s$ on $M$, there is a unique complex structure $\widetilde
J_s$ on $\widetilde M$ that makes $f_0$ into a holomorphic map.  Putting
these together we obtain a family $\widetilde X := \widetilde M \times
S$, $\widetilde X_s = (\widetilde M, \widetilde J_s)$, of complex
manifolds along with a holomorphic map $f : \widetilde X \to X$, $f(x,s)
= f_0(x)$, such that $f_s : \widetilde X_s \to X_s$ is an unramified
finite covering map.
\end{proof}




\begin{theo}
The line bundle $\relt\to S$ is seminegative, negative if the family
$\pi : X \to S$ is effective, and its curvature form is the  
Weil--Petersson metric if the family $\pi : X \to S$ can be polarized.
\end{theo}

\begin{proof}
  The only part to check is the claim on the Weil--Petersson
  metric. If $\omega_s$ is a polarization of $\pi : X \to S$, then
  $f^*\omega_s$ is a polarization of $\widetilde \pi : \widetilde X \to S$, and the
  line bundle defines the Weil--Petersson metric associated to this
  polarization. Given $\alpha \in H^1(X_s,T_{X_s})$ we get an element
$$
f^*\alpha \in H^1(\widetilde X_s, f^*T_{X_s}) = H^1(\widetilde X_s, T_{\smash{\widetilde
X_s}}),
$$
and it is easy to check that
$$
\| f^*\alpha \|^2_{f^*\omega_s} = \|\alpha\|^2_{\omega_s},
$$
so the Weil--Petersson metric on $\pi : X \to S$ is induced by the
one on $\widetilde \pi : X \to S$.
\end{proof}



\subsection*{The relative Kahler cone}


Let again $\pi :\CX \to S$ be a family of compact Kahler manifolds with
zero first Chern class, that we suppose to be effective. Also let $p:
\RKC \to S$ be the relative complexified Kahler cone associated to the
family. A simple consequence of the existence of the Weil--Petersson
metric is that the total space $\RKC$ is a Kahler manifold; see
Corollary~\ref{coro:kahlermanifold}.


Recall that we can view a complex manifold as a smooth manifold equipped
with a complex structure. The set of all complex structures on the
smooth manifold is an infinite-dimensional complex manifold. We can
quotient this manifold by the action of two groups; all diffeomorphisms,
and diffeomorphisms that are homotopic to the identity. Either way we
obtain a topological space that's a finite-dimensional complex manifold
on good days. This is indeed what Kodaira and Spencer did in their
fundamental papers on deformation
theory~\cite{MR0112154,KodairaSpencerIII}.
However they found that the quotient by all diffeomorphisms can be quite
wild, since some manifolds may admit discrete automorphism groups of
infinite order the quotient will not have the structure of a complex
manifold or orbifold. The classical example of this phenomena is the
space of complex structures on a torus \cite[p. 413]{MR0112154}.

In the case of compact Kahler manifolds with zero first Chern class, the
Bogomolov--Tian--Todorov theorem~\cite{Tian} entails that the
Teichm\"uller space -- that is, the space of complex structures
quotiented by diffeomorphisms homotopic to the identity -- is an honest
smooth complex manifold. Problems thus arise only when we try to
quotient by all diffeomorphisms.

People have handled this problem by polarizing the families under
consideration; cf~\cite{MR756781}. This has the effect of throwing away a
number of automorphisms and the resulting quotient then often has the
structure of an honest manifold. We would like to point out that one can
avoid polarizing and quotient by all automorphisms, at least in the case
of manifolds with zero first Chern class. The main step in this
direction is provided by the following result, which we state in our
language:


\begin{theo}[{{\cite[Theorem 12.103]{MR2371700}}}]
Let $\mathcal{C}_\RR$ be the real relative K\"ahler cone over the
Teichm\"uller space. The group $\mathcal{D}^+$ of diffeomorphisms of $M$
acts naturally on $\mathcal{C}_\RR$ and the quotient $\mathcal{C}_\RR /
\mathcal{D}^+$ has the structure of an orbifold.
\end{theo}

From this one can easily deduce that the total space of the relative
complexified K\"ahler cone $\RKC$ has a quotient that admits the
structure of a holomorphic orbifold. The metrics we've considered in
this paper are invariant under the action of automorphisms on each
manifold in a family, so they pass to the quotient and define orbifold
metrics. We must remark that these orbifolds may no longer have the
structure of a fibration over a complex base, since taking the quotient
may destroy any reasonable complex structure on the Teichm\"{u}ller
space.






\bibliographystyle{alpha}
\bibliography{main}

\end{document}
