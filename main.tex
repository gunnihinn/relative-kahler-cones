\documentclass[11pt,a4paper]{amsart}
\usepackage[english]{babel}
\usepackage[T1]{fontenc}
\usepackage[utf8]{inputenc}
\usepackage{lmodern}
\usepackage{amsmath,amsfonts,amssymb}
\usepackage[pdftex]{hyperref}
\usepackage{setspace}
\usepackage{color}
\setstretch{1.1}

\title[The complex geometry of K\"{a}hler cones]
{The complex geometry\\of K\"{a}hler cones}
\author{Gunnar Þór Magnússon}

\newtheorem{theo}{Theorem}[section]
\newtheorem{prop}[theo]{Proposition}
\newtheorem{coro}[theo]{Corollary}
\newtheorem{lemm}[theo]{Lemma}
\newtheorem*{theo*}{Theorem}
\theoremstyle{definition}
\newtheorem*{defi}{Definition}
\newtheorem{exam}[theo]{Example}
\theoremstyle{remark}
\newtheorem*{rema}{Remark}

\newcommand{\QQ}{\mathbb{Q}}
\newcommand{\RR}{\mathbb{R}}
\newcommand{\CC}{\mathbb{C}}
\newcommand{\PP}{\mathbb{P}}

\newcommand{\End}{\mathop{\mathrm{End}}}
\newcommand{\Aut}{\mathop{\mathrm{Aut}}}
\newcommand{\id}{\mathop{\mathrm{id}}}
\newcommand{\Vol}{\mathop{\mathrm{Vol}}}
\def\Im{\mathop{\rm Im}}

\def\ov#1{\overline{#1}}

%% Operators
\DeclareMathOperator{\Gr}{Gr}
\DeclareMathOperator{\Hom}{Hom}
\def\d{\partial}
\def\dbar{\bar\partial}

\def\p{\times}

%% Cohomology groups
\def\coho#1{\mathrm{H}^{#1}}

%% Metric and associated operators
\def\met#1#2{\langle #1, \ov{#2} \rangle}
\def\Lef{\Lambda}

%% Connection and curvature
\def\chern{D}
\def\conn{\nabla}
\def\curv{\tfrac{i}{2\pi}\Theta}

%% Intersection products with a Ahler form
\def\q#1{\frac{1}{\Vol}\int_X #1 \wedge \kf\^{n-1}}
\def\qq#1#2{\frac{1}{\Vol}\int_X #1 \wedge #2 \wedge \kf\^{n-2}}
\def\qqq#1#2#3{\frac{1}{\Vol}\int_X #1 \wedge #2 \wedge #3 \wedge \kf\^{n-3}}
\def\qqqq#1#2#3#4{\frac{1}{\Vol}\int_X #1 \wedge #2 \wedge#3 \wedge#4 \wedge \kf\^{n-4}}

%% The K\"{a}hler form and class of a Hermitian metric
\def\kf{\omega}
\def\ckf{\alpha}

%% Vertical and horizontal vectors
\def\ver#1{#1^{\mathrm{V}}}
\def\hor#1{#1^{\mathrm{H}}}

%% Tangent vectors in K\"{a}hler cone
\def\ton{u}
\def\ttw{v}
\def\tth{z}
\def\tfo{w}

%% Algebra power
\def\^#1{^{[#1]}}

%% K\"{a}hler cones
\def\KC{C}
\def\RKC{\mathcal{\KC}}

\begin{document}

\begin{abstract}
The complexified K\"{a}hler cone of a compact manifold carries a natural
Hermitian metric, given by the intersection product of its cohomology
ring. We show that this metric is K\"{a}hler and calculate its curvature
tensor. We also give a relative version of this metric on the K\"{a}hler
cones of a family of manifolds and show that it extends to a closed
semipositive form on the total space of those cones.
\end{abstract}

\maketitle



\section{On the cone of a fixed manifold}

\subsection*{The K\"{a}hler cone}
Let $X$ be a compact K\"{a}hler manifold of dimension $\dim_{\CC} X = n$.

\begin{defi}
The \emph{K\"{a}hler cone} of $X$ is the set
\begin{equation*}
C_{\RR}(X) = \{ \kf \in \coho{1,1}(X,\RR) 
\mid
\text{$\kf$ contains a K\"{a}hler metric}
\}.
\end{equation*}
The \emph{complexified K\"{a}hler cone} of $X$ is
\begin{align*}
\KC(X) 
&:=
\{ \ckf \in \coho{1,1}(X,\CC) 
\mid
\text{$\Im\ckf$ contains a K\"{a}hler metric}
\}
\\
&\phantom{:}= \coho{1,1}(X,\RR) \oplus i C_{\RR}(X).
\end{align*}
\end{defi}

If there can be no confusion about the underlying manifold $X$, we'll
just write $C$ for its complexified K\"{a}hler cone.
As the names suggest, these are open cones in the finite-dimensional
vector spaces $\coho{1,1}(X,\RR)$ and $\coho{1,1}(X,\CC)$. The K\"{a}hler
cone is the trancendental analogue of the ample cone of a projective
variety. It is described by the the trancendental version of the 
Nakai--Moishezon criteria due to Demailly and Paun~\cite{DemaillyPaun}:


\begin{theo}
The K\"{a}hler cone of $X$ is a connected component of the set of real
$(1,1)$-cohomology classes that are numerically positive on analytic
cycles, that is, classes $\alpha$ such that $\int_{Z} a^p > 0$ for every
irreducible analytic set $Z$ in $X$ of dimension $p$.
\end{theo}


\subsection*{The K\"{a}hler metric on the cone}

Since the complexified K\"{a}hler cone $C$ is an open set in a complex
vector space, we can view it as a complex manifold in its own right.
As such, it carries a natural smooth Hermitian metric, given at each
point $\ckf$ by the inner product that the imaginary part $\Im\ckf$
defines on the cohomology of $X$.

Before we state our first result, let's agree on some notation. If
$\alpha$ is an element of an associative algebra, we write $\alpha\^k
:= \alpha^k/k!$ for all $k \geq 0$. This notation is quite convenient
for calculations with Kahler forms in the cohomology ring of $X$; I
stole it shamelessly from Georg Schumacher.


\begin{prop}
Let $\ton,\ttw$ be elements of $T_C$ at a point $\ckf$. Let $\kf =
\Im\ckf$ and let $\Lef$ be the adjoint of the Lefschetz operator that
$\kf$ defines.  The Hermitian metric on $C$ can be defined in any of the
following ways.
\begin{enumerate}
    \item 
$$
\met{\ton}{\ttw}
= \q{\ton} \cdot \q{\ov{\ttw}}
- \qq{\ton}{\ov\ttw}.
$$
    \item
\hfil
$
\met{\ton}{\ttw}
= \Lef(\ton)\Lef(\ov\ttw)
- \Lef\^2(\ton\wedge\ov\ttw).
$
\hfil
    \item
The metric is the Hermitian form associated to 
$-4\d\dbar\frac{i}{2}\Vol$, where $\Vol : C \to \RR$, $\ckf \mapsto
\Vol(X,\Im\ckf)$.
\end{enumerate}
It follows that this Hermitian metric is a K\"{a}hler metric.
\end{prop}


\begin{proof}
That $(1)$ and $(2)$ define the same Hermitian form is a simple
calculation based on the adjoint property of $\Lef$, and that $(2)$
agrees with the inner product that $\kf$ defines can be seen by taking
the primitive decomposition of $\ton$ and $\ttw$, plugging it into $(2)$
and calculating until the Hodge--Riemann bilinear relations say that we
have the correct inner product.

Note that we can view $\ckf$ as the tautological section $C \to T_C$
associated to the tangent bundle of any open set in a vector space.
Since $\kf = \Im\kf$, we have $\d\kf = \frac{1}{2i} \d\ckf$ and 
$\dbar\kf = -\frac{1}{2i} \dbar\ckf$, where $d = \d + \dbar$ is
the exterior derivative on the vector space $\coho{1,1}(X,\CC)$.
We now have
\begin{equation*}
\displaylines{
-4 \cdot \tfrac{i}{2}\d\dbar \log \Vol
= 
\frac{i}{2}
\q{\d\ckf} \cdot \q{\dbar\ckf}
\hfill\cr\hfill{}
- \frac{i}{2} \qq{\d\ckf}{\dbar\ckf}
}
\end{equation*}
after being careful about sign errors. The Hermitian form associated to
this closed form is clearly $(1)$.
\end{proof}


We record here the basic commutation results on the Lefschetz operator
and its adjoint that we use.


\begin{prop}
Let $\kf$ be a K\"{a}hler class, $L$ the associated Lefschetz operator and
$\Lef$ its adjoint. The following hold for classes of degree $k$:
\begin{itemize}
\item $[L,\Lef] = (k-n)\id$.
\item $[L\^i,\Lef] = (k-n+i-1)L\^{i-1}$.
\item $[L,\Lef\^i] = (k-n+i-1)\Lef\^{i-1}$.
\end{itemize}
\end{prop}


\begin{proof}
The first identity is classical, the second is proved in
\cite[Corollary~1.64]{HuybrechtsGeometry}, and the third follows from
the second by taking adjoints. 
\end{proof}


\begin{rema}
The real version of this metric, defined on the K\"{a}hler cone, has been
studied by Wilson~\cite{Wilson} along with Trenner~\cite{WilsonTrenner},
mostly on the K\"{a}hler cone of manifolds with trivial canonical bundle.
Wilson picked a K\"{a}hler metric to work with and was able to express the
curvature tensor of the metric in terms of differential forms.

In my thesis~\cite{Magnusson} I also studied the real metric by
embedding the K\"{a}hler cone into the space of Hermitian metrics on the
underlying manifold via the Aubin--Calabi--Yau theorem (as in fact
suggested by Wilson in his paper). That space carries a natural
Riemannian metric, known as the Ebin metric when considered on the space
of Riemannian metrics on the underlying smooth manifold;
see~\cite{Ebin,ClarkeRubinstein}. This embedding lets us calculate the
curvature tensor of the metric.

The problem with both approaches is that while the metric on the K\"{a}hler
cone is defined entirely in cohomological terms, the expressions one
gets for its curvature tensor are not cohomological. They involve
differential forms that are induced from harmonic forms on the manifold,
but are themselves neither harmonic nor closed. This makes it difficult
to say anything at all about the curvature of the metric. In this paper
we'll treat the Aubin--Calabi--Yau theorem and actual K\"{a}hler metrics as
red herrings and stay wholly within the cohomology ring of our manifold.

One can also remark that Wilson chooses to work on a subset of the
K\"{a}hler cone; one defined by all K\"{a}hler metrics of a fixed volume. This
makes perfect sense in the real case, since the K\"{a}hler cone is isometric
to the positive real line times such a subset and this restriction lets
us work only with primitive forms. Our metric (or rather its real version)
is the same as Wilson's metric once restricted to this subset.  I don't
see a similar useful complex subspace of the complexified K\"{a}hler cone, so
we work on all of it. This doesn't change anything important.
\end{rema}



\subsection*{The Chern connection and curvature form}

The Chern connection on a holomorphic vector bundle with a Hermitian
metric $h$ is the unique connection that is compatible with the metric
and whose $(0,1)$-part is $\dbar$; that is, it satisfies
$$
d h(\ton, \ov{\ttw}) 
= h(\chern \ton, \ov{\ttw}) + h(\ton, \ov{\chern \ttw}),
\quad
\chern^{0,1} = \dbar
$$
for all sections $u, v$ of the bundle.


\begin{prop}
\label{prop:chernconnection}
The Chern connection of the K\"{a}hler metric on $\KC$ is
$$
\chern \ton
= d\ton 
- \langle \d \kf, \kf \rangle \, \ton
- \langle \ton, \kf \rangle \, \d\kf 
+ \Lambda(\ton \wedge \d\kf).
$$
\end{prop}


\begin{proof}
It's enough to calculate the Chern connection on local holomorphic
vector fields, so let $\ton, \ttw$ be thus. Then the Chern connection
satisfies
$$
\d h(\ton,\ov\ttw)
= h(\chern^{1,0}\ton, \ov\ttw),
$$
and knowing that is enough to describe it all (plus it saves us some
calculations). We have
$$
h(\ton,\ov\ttw)
= \q{\ton} \cdot \q{\ov\ttw} - \qq{\ton}{\ov\ttw}.
$$
First we note that
\begin{equation*}
\displaylines{
\d\q{\ton}  
= 
- \q{\d\kf} \cdot \q{\ton} 
\hfill\cr\hfill
{}+ \q{\d\ton}
{}+ \qq{\ton}{\d\kf}.
}
\end{equation*}
Second, we have
\begin{equation*}
\displaylines{
\d\qq{\ton}{\ov\ttw}
=
{}- \q{\d\kf} \cdot \qq{\ton}{\ov\ttw}
\hfill\cr\hfill
{}+ \qq{\d\ton}{\ov\ttw}
%\cr\hfill
{}+ \qqq{\ton}{\ov\ttw}{\d\kf}.
}
\end{equation*}
Putting these together, we take a deep breath and calculate
\begin{equation*}
\def\bil{\phantom{dh(\ton,\ttw)}}
\displaylines{
\d h(\ton,\ttw)
{}=
\d\biggl( \q{\ton} \cdot \q{\ov\ttw}\biggr)
- \d\biggl( \qq{\ton}{\ov\ttw}\biggr)
\hfill\cr
%\bil{}=
%\d\biggl( \q{\ton} \biggr) \cdot \q{\ov\ttw}
%\hfill\cr\hfill
%{}+ \q{\ton} \cdot \d\biggl( \q{\ov\ttw} \biggr) 
%\cr\hfill
%{}- \d\biggl( \qq{\ton}{\ov\ttw}\biggr)
%\cr
\bil
{}= 
\hfill
\textcolor{green}{- \q{\d\kf} \cdot \q{\ton} \cdot \q{\ov\ttw}}
%\quad(1)
\cr\hfill
\textcolor{red}{{}+ \q{\d\ton}\cdot \q{\ov\ttw}}
%\quad(2)
\cr\hfill
{}+ \qq{\ton}{\d\kf}\cdot \q{\ov\ttw}
%\quad(3)
\cr\hfill
\textcolor{blue}{{}- \q{\ton} \cdot \q{\d\kf} \cdot \q{\ov\ttw}}
%\quad(4)
\cr\hfill
\textcolor{blue}{{}+ \q{\ton} \cdot \qq{\ov\ttw}{\d\kf}}
%\quad(5)
\cr\hfill
\textcolor{green}{{}+ \q{\d\kf} \cdot \qq{\ton}{\ov\ttw}}
%\quad(6)
\cr\hfill
\textcolor{red}{{}- \qq{\d\ton}{\ov\ttw}}
%\quad(7)
\cr\hfill
{}- \qqq{\ton}{\d\kf}{\ov\ttw}.
%\quad(8)
\cr
\bil
{}=
\textcolor{red}{h(\d\ton, \ov\ttw)}
\textcolor{green}{- h(\d\kf, \kf) \cdot h(\ton, \ov\ttw)}
\textcolor{blue}{- h(\ton, \kf) \cdot h(\d\kf, \ov\ttw)}
\hfill
\cr\hfill
{}+ \qq{\ton}{\d\kf}\cdot \q{\ov\ttw}
\cr\hfill
{}- \qqq{\ton}{\d\kf}{\ov\ttw}.
}
\end{equation*}
By Lemma~\ref{lemm:hodgeproduct} there exists a unique vector $\ton
\star \d\kf \in \coho{1,1}(X,\CC)$ such that
\begin{align*}
\ton \star \d\kf \wedge \kf\^{n-2}
&= \ton \wedge \d\kf \wedge \kf\^{n-3},
\\
\tfrac{n-1}{n-2} \ton \star \d\kf \wedge \kf\^{n-1}
&= \ton \wedge \d\kf \wedge \kf\^{n-2}.
\end{align*}
Then we have
\begin{equation*}
\displaylines{
\frac{n-1}{n-2} 
\q{\ton \star \d\kf}\cdot\q{\ov\ttw}
- \qq{\ton \star \d\kf}{\ov\ttw}.
\hfill\cr\hfill{}
=\tfrac{1}{n-2} 
h(\ton \star \d\kf, \kf) \cdot h(\kf, \ov\ttw)
+ h(\ton \star \d\kf, \ov\ttw)
}
\end{equation*}
Putting this all together, we get
\begin{align*}
\chern \ton 
&= d\ton 
- h(\d \kf, \kf)\, \ton
- h (\ton, \kf)\, \d\kf 
+ \tfrac{1}{n-2} h(\ton \star \d\kf, \kf)\, \kf
+ \ton \star \d\kf
\\
&= d\ton 
- h(\d \kf, \kf)\, \ton 
- h (\ton, \kf)\, \d\kf 
+ \tfrac{1}{n-1} \Lambda\^{2}(\ton \wedge \d\kf)\, \kf
\\
&
\qquad \qquad \qquad \qquad
\qquad \qquad \qquad \qquad
+ \Lambda(\ton \wedge \tth) 
- \tfrac{1}{n-1} \Lambda\^{2} (\ton\wedge\tth)\, \kf
\\
&= d\ton 
- h(\d \kf, \kf)\, \ton 
- h (\ton, \kf)\, \d\kf 
+ \Lambda(\ton \wedge \d\kf),
\end{align*}
by using the expression of $\ton \star \d\kf$ from
Lemma~\ref{lemm:hodgeproduct}.
\end{proof}


\begin{lemm}
\label{lemm:hodgeproduct}
Let $\ton, \tth$ be $(1,1)$-classes on $X$. Then there is a unique
$(1,1)$-class $\ton \star \tth$ on $X$ such that
\begin{align*}
\ton \star \d\kf \wedge \kf\^{n-2}
&= \ton \wedge \d\kf \wedge \kf\^{n-3},
\\
\tfrac{n-1}{n-2} \ton \star \d\kf \wedge \kf\^{n-1}
&= \ton \wedge \d\kf \wedge \kf\^{n-2}.
\end{align*}
In fact,
\begin{equation*}
\ton \star \tth
= \Lambda(\ton \wedge \tth) 
- \tfrac{1}{n-1} \Lambda\^{2} (\ton\wedge\tth)\, \kf.
\end{equation*}
\end{lemm}


\begin{proof}
Let $L x = x \wedge \kf$ be the Lefschetz operator on $\coho{*}(X,\CC)$.
The hard Lefschetz theorem says that $L\^{n-2} : \coho{1,1}(X,\CC) \to
\coho{n-1,n-1}(X,\CC)$ is an isomorphism. Since $\ton \wedge \tth \wedge
\kf\^{n-3} = L\^{n-3}(\ton \wedge \tth)$ is an $(n-1,n-1)$-class, the
class $\ton \star \tth$ exists, is unique and equals
$(L\^{n-2})^{-1}(L\^{n-3}(\ton \wedge \tth))$.

Let $\Lambda$ be the adjoint of the Lefschetz operator. Recall that
$$
[L\^i, \Lambda] = (k - n + i - 1) L\^{i-1}
$$
on the space of $k$-classes on $X$; see
\cite[Corollary~1.64]{HuybrechtsGeometry}. On the space of $4$-classes
with $i = n-2$ this gives $[L\^{n-2}, \Lambda] = L\^{n-3}$, so
$$
L\^{n-3}(\ton \wedge \tth)
= L\^{n-2}\Lambda(\ton \wedge \tth) 
- \Lambda L\^{n-2}(\ton \wedge \tth).
$$
The first term is fine, so let's focus on the second one. We make the
banal remark that $\Lambda\^2(\ton \wedge \tth)$ is a scalar and
calculate in succession that
\begin{align*}
L\^{n-2}(\ton \wedge \tth) 
&= \Lambda\^{2}(\ton \wedge \tth)\, \kf\^n,
\\
\Lambda L\^{n-2}(\ton \wedge \tth) 
&= \Lambda\^{2}(\ton \wedge \tth)\, \kf\^{n-1},
\\
(\Lambda\^{n-2}a)^{-1}(\Lambda L\^{n-2}(\ton \wedge \tth) )
&= \tfrac{1}{n-1} \Lambda\^{2}(\ton \wedge \tth)\, \kf.
\end{align*}
This gives the first of our results. For the second we simply calculate
\begin{align*}
\ton \star \d\kf \wedge \kf\^{n-1}
= \frac{1}{n-1} \ton \star \d\kf \wedge \kf\^{n-2} \wedge \kf
&= \frac{1}{n-1} \ton \wedge \d\kf \wedge \kf\^{n-3} \wedge \kf
\\
&= \frac{n-2}{n-1} \ton \wedge \d\kf \wedge \kf\^{n-2},
\end{align*}
which is equivalent to the second result.
\end{proof}


\begin{lemm}
\label{coro:kahlerform}
$\chern^{1,0} \kf = -\d\kf$.
\end{lemm}


\begin{proof}
Let $\ton$ be a tangent field on $\KC$ and consider the equation
$\langle \ton, \kf \rangle = \Lef \ton$. Upon differentiating this in
the direction of $\ttw$ we get
$$
\langle \chern_{\ttw}\ton, \kf \rangle 
+ \langle \ton, \chern_{\ttw}\kf \rangle 
= (d_{\ttw}\Lef) \ton
+ \Lef d_{\ttw}\ton
= \tfrac{1}{2}i \langle \ton, \ttw\rangle
+ \Lef d_{\ttw}\ton.
$$
Proposition~\ref{prop:chernconnection} now gives
$$
\displaylines{
\langle \chern_{\ttw}\ton, \kf \rangle 
=
\Lef(d_{\ttw}\ton)
- i\Lef\ton\cdot \Lef\ttw
+ i\Lef\^2(\ton \wedge \ttw)
}
$$
so
\begin{align*}
\langle \ton, \chern_{\ttw}\kf \rangle
&=
-\tfrac{1}{2}i \Lef\ton \cdot \Lef \ttw
+\tfrac12 i \Lef\^2(\ton \wedge \ttw)
+ i\Lef\ton\cdot \Lef\ttw
- i\Lef\^2(\ton \wedge \ttw)
\\
&=
\langle \ton, -\tfrac{1}{2i} \ttw \rangle
= \langle \ton, -\d\kf(\ttw) \rangle.
\qedhere
\end{align*}
\end{proof}


\begin{theo}
The curvature tensor of the K\"{a}hler metric on $\KC$ is
\begin{equation*}
R(\ton,\ov\ttw,\tth,\ov\tfo)
= - \tfrac14 h(\ton,\ov\ttw)\, h(\tth,\ov\tfo)
- \tfrac14 h (\ton,\ov\tfo)\, h(\tth,\ov\ttw)
+ \tfrac14 \langle \ton \wedge \tth, \ov{\ttw\wedge\tfo} \rangle.
\end{equation*}
\end{theo}


\begin{proof}
Let $\ton$ be a local holomorphic section of the tangent bundle of $\KC$.
By definition, we have $\curv = \chern^2$ and since our connection is
Hermitian, we get $\curv = \dbar \chern$ on holomorphic sections.
Note that since $\kf = \Im \alpha$, we have $\d\dbar \kf = 0$.
Proposition~\ref{prop:chernconnection} and
Lemma~\ref{coro:kahlerform} now give
\begin{align*}
\dbar_{\ov\tfo} \chern_{\tth} \ton
&=
- \tfrac{1}{2i} \langle \tth,\ov{\chern^{1,0}_{\tfo}\kf} \rangle\, \ton
- \tfrac{1}{2i} \langle \ton, \ov{\chern^{1,0}_{\tfo}\kf} \rangle\, \tth
+ \tfrac{1}{2i} \dbar(\Lambda(\ton \wedge \tth))
\\
&=
- \tfrac{1}{4} \langle \tth,\ov{\tfo} \rangle\, \ton
- \tfrac{1}{4} \langle \ton, \ov{\tfo} \rangle\, \tth
+ \tfrac{1}{2i} \dbar(\Lambda(\ton \wedge \tth)).
\end{align*}
Upon taking the inner product with $\ttw$, we then get
$$
R(\ton,\ov\ttw,\tth,\ov\tfo)
= -\tfrac14
\langle \ton, \ov\ttw \rangle
\langle \tth,\ov{\tfo} \rangle
-\tfrac14
\langle \ton, \ov{\tfo} \rangle
\langle \tth, \ov\ttw \rangle
+ \tfrac{1}{4} \langle \ton \wedge \tth, \ov{\ttw \wedge \tfo}\rangle.
\qedhere
$$
\end{proof}


\begin{coro}
The derived curvature tensors are as follows.

\smallskip
\noindent
$\bullet$
Holomorphic bisectional curvature:
$$
B(\ton,\ov\ttw) 
= -\tfrac14 |\ton|^2 |\ttw|^2
- \tfrac14 |\met{\ton}{\ttw}|^2
+ \tfrac14 |\ton \wedge \ttw|^2
$$
\smallskip
\noindent
$\bullet$
Holomorphic sectional curvature:
$$
H(\ton) 
=
-\tfrac12 |\ton|^2
+\tfrac14 |\ton \wedge \ton|^2
$$
\smallskip
\noindent
$\bullet$
Ricci curvature:
$$
r(\ton, \ov\ttw) 
=
-\tfrac14(n+1) \met{\ton}{\ttw}
+ \sum_{j=1}^{h^{1,1}} \met{\ton\wedge x_j}{\ttw \wedge x_j},
$$
where $(x_1,\ldots,x_{h^{1,1}})$ is an orthonormal basis.
\end{coro}




\subsection*{Completeness}

The theorem of Demailly and Paun describes the boundary of the
K\"{a}hler cone of a compact complex manifold. It consists of three
parts:
\begin{enumerate}
\item Limits of classes $a_t$ whose volume $\int_X a_t\^n$
tends to zero.
\item Limits of classes whose volume tends to infinity.
\item Limits of classes whose volume tends to some positive real
number, but there exists a proper irreducible complex subspace $Z
\subset X$ of dimension $p \geq 1$ whose volume tends to zero.
\end{enumerate}
Complexifying the K\"{a}hler cone adds one more entry to this list:
\begin{enumerate}
\item[(4)] Limits of classes that fall into none of cases (1)--(3), but
whose real part tends to infinity.
\end{enumerate}

Let us conspire to call $\mathcal{P} := \{\alpha \in \coho{1,1}(X,\CC)
\mid (\Im\alpha)^n > 0\}$ the cone of complexified volume classes on
$X$, or the complexified volume cone. It contains the complexified
K\"{a}hler cone, but is in almost all cases bigger than it.

\begin{prop}
\label{prop:fofo}
The metric on the complexified K\"{a}hler cone of $X$ is complete if and
only if cone is a connected component of the complexified volume cone.
\end{prop}

\begin{proof}
We first show that the classes on the first two parts of the boundary
pose no problems. Let $I$ be an interval in the real numbers and let
$\gamma : I \to \KC$ be a smooth path in $\KC$ that approaches the
boundary of $\KC$. Let $I_m = [a, b_m]$ be an increasing exhaustion
of $I$ by compact intervals and let $\gamma_m$ be the restriction
of $\gamma$ to $I_m$. Suppose that the volume $\Vol(X,\Im\gamma_m)$
tends to either zero or infinity as $m$ tends to infinity.

\begin{lemm}
Let $I = [a,b]$ be a compact interval in the real numbers $\RR$,
and let $\gamma : I \to \KC$ be a smooth path. The length of
the path $\gamma$  satisfies
$$
L(\gamma) \geq
\frac{\sqrt 2}{\sqrt n}
\bigl| \log \Vol(X,\Im\gamma(b))
- \log \Vol(X,\Im\gamma(a))
\bigr|.
$$
\end{lemm}

\begin{proof}[Sketch of proof.]
We apply the Cauchy--Schwarz inequality to the scalar product
$h(\ton,\kf)$; this gives
$$
|\ton \cdot \log \Vol(X,\kf)|^2 
= |\tfrac{1}{2i}h(\ton,\kf)|^2 \leq \tfrac{n}{2} h(\ton,\ov\ton).
$$
Integrating and applying the triangle inequality then gives the
announced estimate.
\end{proof}

Applying the lemma on each interval $I_m$ then gives that
\begin{equation*}
  L(\gamma) = \lim\limits_{m \to +\infty} L(\gamma_m) = +\infty.
\end{equation*}
Thus the limit class $\lim \gamma(t)$ on the boundary cannot be
approached by paths in $\KC$ of finite length.

If the complexified K\"{a}hler and volume cones of $X$ do not
coincide, then there exists a class $\alpha$ on the boundary of
$\KC_{\RR}$ such that $\Vol(X,\alpha) > 0$, but there is a proper
complex subspace $Z \subset X$ such that $\Vol(Z,\alpha) = 0$.

As $\alpha$ is on the boundary of the K\"{a}hler cone, then there
exists a K\"{a}hler class $\kf$ such that $\gamma(t) := \alpha +
t\kf$ is in the K\"{a}hler cone for all $t > 0$. The tangent vectors
of the path $\gamma$ are $\gamma'(t) = \kf$, and the norm of
$\gamma'(t)$ at the point $\gamma(t)$ is
$$
\displaylines{
  h(t) :=
  g(\gamma'(t), \gamma'(t))(\gamma(t)) =
  \left(
    \frac{1}{\Vol(X,\gamma(t))}
    \int_X \kf \wedge (\alpha + t\kf)\^{n-1}
  \right)^2
\hfill\cr\hfill
{}- \frac{1}{\Vol(X,\gamma(t))}
    \int_X \kf^2 \wedge (\alpha + t\kf)\^{n-2}.
}
$$
Each of these integrals, and the function $t \mapsto \Vol(X,\gamma(t))$,
is a polynomial in $t$ on some small interval $[0,t_0]$. As $\lim_{t\to
0} \Vol(X,\gamma(t)) > 0$ the function $t \mapsto h(t)$ is continuous
and positive on a compact interval, so the integral $L(\gamma)$ of its
square root exists and is finite.

Finally suppose that we have a path $\gamma : I = [a,b] \to C$ that
doesn't fit into cases (1)--(3) but meanders along the real direction in
the complexified K\"{a}hler cone. I then claim that the set
$$
J := \{ \Im \gamma(t) \mid t \in I \}
$$
is relatively compact in $C_{\RR}$: Let $\nu$ be a limit point of that
set. Since the volumes of the classes $\Im \gamma(t)$ neither tend
towards $0$ nor $\infty$ they are contained in some compact interval, so
the volume of $\nu$ is positive. Similarly, we've excluded that the
volume of some subvariety $Z$ with respect to $\nu$ is zero by
hypothesis, so $\Vol(Z,\nu) > 0$ for all subspaces $Z$ of $X$. Thus
$\nu$ is a K\"{a}hler class. 
This implies that
$$
%m(t) := \inf_{\nu \in J} h(\gamma'(t),\ov{\gamma'(t)})_{\nu}
%\quad\text{and}\quad
M(t) := \sup_{\nu \in J} h(\gamma'(t),\ov{\gamma'(t)})_{\nu}
$$
exists, so the length of $\gamma$ can be bounded by above by
$\int_I M(t) dt$.
\end{proof}

\subsection*{Pullbacks}

A holomorphic map $f : X \to Y$ between compact K\"{a}hler manifolds induces
a morphism $f^* : \coho{*}(Y,\CC) \to \coho{*}(X,\CC)$ in cohomology
that respects the Hodge decomposition. However, if $\kf$ is a K\"{a}hler
class on $Y$, then $f^*\kf$ is hardly ever a K\"{a}hler class on $X$. This
happens mostly if $f$ is either an embedding or a finite covering map.

\begin{prop}
Let $f : X \to Y$ be a finite surjective morphism. Let $h_X$ and $h_Y$
be the K\"{a}hler metrics on the complexified K\"{a}hler cones of $X$ and
$Y$, respectively. Then the pullback morphism $f^* : \KC(Y) \to \KC(X)$
is a Hermitian embedding.
\end{prop}

\begin{proof}
Let $\ckf$ be a point in $\KC(Y)$ and $\kf = \Im \ckf$. The volume of
$X$ with respect to $f^*\kf$ is
\begin{equation*}
  \Vol(X,f^*\kf) = p \, \Vol(Y,\kf)
\end{equation*}
as $f$ is finite of degree $p$. It follows that $f^*$ is an embedding.
\end{proof}

\begin{coro}
The group $\Aut X$ of holomorphic automorphisms of $X$ acts by
isometries on the K\"{a}hler cone $\KC(X)$.
\end{coro}

A closer look reveals that this last statement contains less information
than first meets the eye. The automorphism group $\Aut X$ of a compact
complex manifold is a Lie group and it splits roughly into two parts; a
positive-dimensional group given by the flows of holomorphic vector
fields, or elements of $\coho{0}(X,T_X)$, and a discrete part consisting of
``other'' automorphisms. The isomorphisms generated by vector fields act
trivially on the cohomology ring of $X$, so the only part of $\Aut X$
that possibly acts by nontrivial isometries on $\KC(X)$ is discrete.



\bibliographystyle{alpha}
\bibliography{main}

\end{document}
